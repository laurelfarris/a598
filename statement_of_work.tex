\documentclass[12pt]{article}
\usepackage[margin=1in]{geometry}
\pagenumbering{gobble}
\setlength{\parindent}{0em}
\setlength{\parskip}{0.75em}

\title{ASTR 598 - Statement of Work}
\date{Spring 2016}
\author{Laurel Farris}

\begin{document}
\maketitle

\large\textbf{Topic}\normalsize

The topic of this course will be Coronal Seismology. This is
rather broad, which should help to put my previous research into context,
as well as reveal other possible thesis topics.

\large\textbf{Coursework}\normalsize

The four points of emphasis as laid out by the syllabus for ASTR
598 are addressed as follows:
\begin{enumerate}
    \item \textbf{Literature review capabilities:} This will be the main
        component of the course. I will read at least one review article
        each week. I will continuously update the writeup, and
        possibly the presentation (both described below)
        to develop this habit permanently.
    \item \textbf{Scientific writing skills:} I will complete a
        writeup describing the
        background information from the literature, general
        relevance to science/astronomy, possible projects, and how the
        research I have done so far ties into the main topic of coronal
        seismology. I will probably use the \texttt{aastex} style.
    \item \textbf{Scientific presentation skills:} I will present the results
        of this course at a pizza lunch talk on April 25th at 12:30pm.
    \item \textbf{Experience with one or more research investigation tools:}
        As most of the work I have done so far has involved data analysis, this
        part will comprise the smallest bit of the time spent specifically for
        this course, though I will continue to work on it for my regular
        research.
\end{enumerate}

\large\textbf{Milestones}\normalsize

\begin{itemize}
    \item 2/4/16 Read Nakariakov: ``Coronal Waves and Oscillations'' and
        choose another review article to read.
    \item 2/11/16 Read the chosen article from above, review wave concepts,
        and re-read the Nakariakov article. Also develop a way to stay
        organized for the course, as I have already lost my to-do list.
\end{itemize}

\large\textbf{Schedule}\normalsize

We will meet every Thursday at 11am to discuss the assigned reading
for that week, ensure that I am still on track, and make any necessary
changes or additions to the statement of work.
Outside of this meeting, I plan on spending at least eight
hours a week addressing the points listed above.



\end{document}
