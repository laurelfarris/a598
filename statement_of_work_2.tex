\documentclass[12pt]{article}
\usepackage[margin=1in]{geometry}
\pagenumbering{gobble}
\setlength{\parindent}{0em}
\setlength{\parskip}{0.75em}
\usepackage{enumitem}
\setlist[1]{itemsep=-2pt}%,topsep=-2pt}
\usepackage{hyperref}
\usepackage{color}
\usepackage{multicol}

\usepackage{titlesec}
\titleformat*{\section}{\large\bfseries}

\definecolor{cobalt}{rgb}{0.0,0.28,0.67}

\begin{document}

\begin{centering}
    \Large\textbf{ASTR 598 - Statement of Work\\}
    \vspace{0.5em}
    \large\textbf{Laurel Farris\\}
    \vspace{0.25em}
    \large\textbf{Spring 2016\\}
    \vspace{0.25em}
\end{centering}


\large\textbf{Topic}\normalsize

The topic of this course will be Coronal Seismology. This is
rather broad, which should help to put my previous research into context,
as well as reveal other possible thesis topics.
The main areas of coronal seismology to be addressed throughout the
semester are:
\begin{enumerate}
    \item Kink oscillations
    \item Sausage oscillations
    \item Acoustic oscillations
    \item Propagating acoustic waves
    \item Propagating fast waves
    \item Torsional modes
    \item Mixed modes
\end{enumerate}

\large\textbf{Coursework}\normalsize

The four points of emphasis as laid out by the syllabus for ASTR
598, along with the portion of allotted time to be spent on them
are addressed as follows:
\begin{enumerate}
    \item \textbf{30\% Literature review capabilities:} This will be the main
        component of the course. I will read at least one review article
        each week. I will continuously update the writeup, and
        possibly the presentation (both described below)
        to develop this habit permanently.
    \item \textbf{30\% Scientific writing skills:} I will complete a
        writeup describing the
        background information from the literature, general
        relevance to science/astronomy, possible projects, and how the
        research I have done so far ties into the main topic of coronal
        seismology. I will probably use the \texttt{aastex} style.
    \item \textbf{30\% Scientific presentation skills:}
        I will present the results
        of this course at a pizza lunch talk on April 25th at 12:30pm.
    \item \textbf{10\% Experience with one or more research investigation
    tools:}
        As most of the work I have done so far has involved data
        analysis, this
        part will comprise the smallest bit of the time spent
        specifically for
        this course, though I will continue to work on it for my regular
        research. This portion will primarily consist of adding my
        methods, scientific question, etc. to the writeup and
        connecting to the topic for this course.
\end{enumerate}

\large\textbf{Grading}\normalsize

The grading scheme for this course will be based on the fraction of
time dedicated to each of the four tasks listed above (i.e., each
portion contributes the specified percentage to my grade).
100\% completion will be an A, 80\% will be a B, and 60\% will be a C.


\large\textbf{Milestones}\normalsize

We will meet every Thursday at 11am to discuss the assigned reading
for that week, ensure that I am still on track, and make any necessary
changes or additions to the statement of work.
Outside of this meeting, I plan on spending at least eight
hours a week addressing the points listed above.
At the time of writing (2/4/2016) my plan is to read two papers a
week, and add to both the writeup and the presentation. This may be
changed to a two week block, with the reading taking place one week
and the writing taking place the next.
A detailed schedule is including on the next page (to be revised and
adjusted as
needed throughout the semester).

\newpage
\section*{1/28}
%\begin{multicols}{2}
\vspace{-0.5cm}
\begin{itemize}
    \item Discuss syllabus
    \item Set up a github account for a598
    \item Set up aastex for writeup
    \item Write a `Statement of Work'
    \item Select another review article
\end{itemize}
%\end{multicols}

\section*{2/04}
\vspace{-0.5cm}
\begin{itemize}
    \item Add a complete schedule to the statement of work
    \item Read section 2 in Nakariakov
    \item Read Ch. 6 on MHD in Aschwanden
    \item Start formulating a list of $\sim$6 topics
            for the semester
\end{itemize}

\section*{2/11}
\vspace{-0.5cm}
\begin{itemize}
    \item Read Ch. 7 on MHD Oscillations in Ashwanden
    \item Read section 2 in Nakariakov
    \item Read 
        \href{http://cdsads.u-strasbg.fr/abs/2015A\%26A...578A..99P}
        {\textcolor{cobalt}{CORONAL LOOP OSCILLATIONS OBSERVED WITH THE
        \emph{TRANSITION REGION AND CORONAL EXPLORER}}}
        \\-Aschwanden et al.
    \item Read
        \href{http://iopscience.iop.org/article/10.1086/307502/meta}
        {\textcolor{cobalt}{Excitation and damping of broadband
        kink waves in the solar corona}}\\-D. J. Pascoe et al.
    \item Summarize papers in writeup
    \item Create presentation slide for papers
    \item Add a description of research to writeup
\end{itemize}

\vspace{-0.5cm}
\section*{2/18}
\vspace{-0.5cm}
\begin{itemize}
    \item Read [paper 1] on Sausage Modes
    \item Read [paper 2] on Sausage Modes
    \item Summarize papers in writeup
    \item Create presentation slide for papers
\end{itemize}

\section*{2/25}
\vspace{-0.5cm}
\begin{itemize}
    \item Read [paper 1] on topic 3
    \item Read [paper 2] on topic 3
    \item Summarize papers in writeup
    \item Create presentation slide for papers
\end{itemize}

\section*{3/03}
\vspace{-0.5cm}
\begin{itemize}
    \item Read [paper 1] on topic 4
    \item Read [paper 2] on topic 4
    \item Summarize papers in writeup
    \item Create presentation slide for papers
\end{itemize}

\section*{3/10}
\vspace{-0.5cm}
\begin{itemize}
    \item Read [paper 1] on topic 5
    \item Read [paper 2] on topic 5
    \item Summarize papers in writeup
    \item Create presentation slide for papers
\end{itemize}

\section*{3/17}
\vspace{-0.5cm}
\begin{itemize}
    \item Spring Break
\end{itemize}

\section*{3/24}
\vspace{-0.5cm}
\begin{itemize}
    \item Read [paper 1] on topic 6
    \item Read [paper 2] on topic 6
    \item Summarize papers in writeup
    \item Create presentation slide for papers
\end{itemize}

\section*{3/31}
\vspace{-0.5cm}
\begin{itemize}
   \item Read [paper 1] on topic 7
   \item Read [paper 2] on topic 7
   \item Summarize papers in writeup
   \item Create presentation slide for papers
\end{itemize}

\section*{4/07}
\vspace{-0.5cm}
\begin{itemize}
    \item Complete draft of writeup
    \item Complete draft of presentation
\end{itemize}

\section*{4/14}
\vspace{-0.5cm}
\begin{itemize}
    \item Discuess Revisions for paper and presentation
    \item  Revise writeup
    \item Revise presentation
    \item Practice presentation
\end{itemize}

\section*{4/21}
\vspace{-0.5cm}
\begin{itemize}
    \item Prepare for presentation on Monday
\end{itemize}

\end{document}
