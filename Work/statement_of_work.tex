\documentclass[12pt]{article}
\usepackage[margin=1in]{geometry}
\setlength{\parindent}{0em}
\setlength{\parskip}{0.5em}
%\setlength{\parskip}{\baselineskip}
\usepackage{enumitem}
\setlist[1]{itemsep=-2pt}
\usepackage{titlesec}
\titlespacing{\section}{0pt}{\parskip}{-\parskip}
\titlespacing{\subsection}{0pt}{\parskip}{-\parskip}
\titlespacing{\subsubsection}{0pt}{\parskip}{-\parskip}
\usepackage{color}

\definecolor{cobalt}{rgb}{0.0,0.28,0.67}
\usepackage{hyperref}
\hypersetup{
    colorlinks=true,
    urlcolor=cobalt,
}
\urlstyle{same}


\begin{document}
\begin{center}\Large
    \Large ASTR 598 - Statement of Work\\
    \large Laurel Farris\\
    \normalsize Spring 2016
\end{center}

\subsection*{Topic}
The topic of this course is Coronal Seismology. This is rather
broad, which should help
to put my previous research into context, as well as reveal other
possible thesis topics. The
main areas of coronal seismology to be addressed throughout the
semester are:
\begin{enumerate}
    \item Kink oscillations
    \item Sausage oscillations
    \item Acoustic oscillations
    \item Propagating acoustic waves
    \item Propagating fast waves
    \item Torsional modes
    \item Mixed modes
\end{enumerate}

\subsection*{Coursework}
The four points of emphasis as laid out by the syllabus for ASTR 598,
along with the portion
of allotted time to be spent on them are addressed as follows:
\begin{enumerate}
    \item \textbf{30\% Literature review capabilities:}
        Each week I will choose two papers to read and discuss the following
        week. One will be either a reference from a review article,
        a general overview of the topic, or something covering the first
        observations of the phenomenon. The other will be something fairly
        recent to see how knowledge of the topic has developed and how it is
        being studied now.
    \item \textbf{30\% Scientific writing skills:}
        I will complete a writeup, \texttt{aastex} style,
        describing the background
        information from the literature, general relevance to
        science/astronomy, possible projects,
        and my current research. Each paper will be summarized in
        $\sim$ 1/3 of a page.
    \item \textbf{30\% Scientific presentation skills:}
        The results of this course will be presented at a pizza lunch
        talk on April 25th at 12:30 pm.
    \item \textbf{10\% Experience with one or more research
        investigation tools:}
        As most of the work I have done so far has involved data
        analysis, this part will comprise the smallest portion of the
        total workload. A short description of my research will be provided,
        along with how it ties in with the main topic.
\end{enumerate}

\newpage
\subsection*{Grading}
The grading scheme for this course will be based on the fraction of
time dedicated to each
of the four tasks listed above (i.e., each portion contributes the
specified percentage to my grade). 100\% completion will be an A,
80\% will be a B, and 60\% will be a C.

\subsection*{Milestones}
We will meet every Thursday at 11am to discuss the assigned reading
for that week, ensure that I am still on track, and make any necessary
changes or additions to the statement of
work. Outside of this meeting, I plan on spending at least eight hours
a week addressing the points listed above. At the time of writing
(2/4/2016) my plan is to read the two chosen papers every
week, and add to both the writeup and the presentation.
A detailed schedule is included on the next page
(to be revised and adjusted as needed throughout the semester).

\newpage
\subsection*{Schedule}
\subsubsection*{1/28}
\begin{itemize}
    \item Discuss syllabus
    \item Set up a github account for a598
    \item Set up aastex for writeup
    \item Write a Statement of Work
    \item Select another review article
\end{itemize}

\subsection*{2/04}
\begin{itemize}
    \item Add a complete schedule to the statement of work
    \item Read section 2 in Nakariakov
    \item Read Ch. 6 on MHD in Aschwanden
    \item Start formulating a list of topics for the semester
\end{itemize}

\subsection*{2/11}
\begin{itemize}
    \item Read Ch. 7 on MHD Oscillations in Ashwanden
    \item Read section 2 in Nakariakov
    \item Read
        \href{http://adsabs.harvard.edu/abs/1999ApJ...520..880A}
        {\textcolor{cobalt}{CORONAL LOOP OSCILLATIONS OBSERVED WITH THE TRANSITION
        REGION AND CORONAL EXPLORER}}\\
        -Aschwanden et al.
    \item Read
        \href{http://cdsads.u-strasbg.fr/abs/2015A\%26A...578A..99P}
        {\textcolor{cobalt}{Excitation and damping of broadband
        kink waves in the solar corona}}\\
        -D. J. Pascoe et al.
    \item Summarize papers in writeup
    \item Create presentation slide for papers
    \item Add a description of research to writeup
\end{itemize}

\subsection*{2/18}
\begin{itemize}
    \item Read
        \href{http://adsabs.harvard.edu/abs/2015ApJ...810...87L}
        {\textcolor{cobalt}{Sausage Waves in Transversely
        Nonuniform Monolithic Coronal Tubes}}\\
        -Lopin and Nagorny
    \item Read
        \href{http://cdsads.u-strasbg.fr/abs/2011ApJ...729L..18M}
        {\textcolor{cobalt}{Observations of Sausage Modes in Magnetic Pores}}\\
        -Morton et al.
    \item Summarize papers in writeup
    \item Create presentation slide for papers
\end{itemize}

\subsection*{2/25}
\begin{itemize}
    \item Read
        \href{http://cdsads.u-strasbg.fr/abs/2010MNRAS.405.2317S}
        {\textcolor{cobalt}{Observations from \emph{Hinode}/EIS of
        intensity oscillations above a bright point: signature of the
        leakage of acoustic oscillations in the inner corona}}\\
        -A. K. Srivastava and B. N. Dwivedi
    \item Read
        \href{http://cdsads.u-strasbg.fr/abs/1984ApJ...279..857R}
        {\textcolor{cobalt}{On Coronal Oscillations}}\\
        -Roberts, B.; Edwin, P. M.; Benz, A. O.
    \item Summarize papers in writeup
    \item Create presentation slide for papers
\end{itemize}

\subsection*{3/03}
\begin{itemize}
    \item \url{http://search.proquest.com/docview/1548708487}
    \item Read
        \href{http://cdsads.u-strasbg.fr/abs/2001A\%26A...370..591R}
        {\textcolor{cobalt}{Slow magnetoacoustic waves in coronal loops:
        EIT and TRACE}}\\
        -Robbrecht et al.
    \item Read
        \href{http://cdsads.u-strasbg.fr/abs/2013ApJ...778...26U}
        {\textcolor{cobalt}{Measuring Temperature-dependent
        Propagating Disturbances in Coronal Fan Loops
        Using Multiple SDO/AIA Channels and the Surfing Transform Technique}}\\
        -Uritsky et al.
    \item Summarize papers in writeup
    \item Create presentation slide for papers
\end{itemize}

\subsection*{3/10}
\begin{itemize}
    \item Read \href{http://cdsads.u-strasbg.fr/abs/2013A\%26A...554A.144Y}
        {\textcolor{cobalt}{Distinct propagating fast wave trains associated
        with flaring energy releases}}\\
         -Yuan, D. et al.
     \item Read \href{http://cdsads.u-strasbg.fr/abs/2012ApJ...745L..18A}
        {\textcolor{cobalt}{First Simultaneous Observation of an H$\alpha$
        Moreton Wave, EUV Wave, and Filament/Prominence Oscillations}}\\
        -Asai, et al.
    \item Summarize papers in writeup
    \item Create presentation slide for papers
\end{itemize}

\subsection*{3/17}
\begin{itemize}
    \item Spring Break
\end{itemize}

\subsection*{3/24}
\begin{itemize}
    \item Read \href{http://cdsads.u-strasbg.fr/abs/2013ApJ...767...16V}
        {Coronal Alfv\'en Speed Determination:
        Consistency between Seismology Using AIA/SDO Transverse Loop
        Oscillations and Magnetic Extrapolation}\\
        -Verwichte, E.; Van Doorsselaere, T.; Foullon, C.; White, R. S.
    \item Read \href{http://cdsads.u-strasbg.fr/abs/2012ApJ...759..144T}
        {Persistent Doppler Shift Oscillations
                Observed with Hinode/EIS in the Solar Corona:
                Spectroscopic Signatures of Alfv\'enic Waves and Recurring Upflows}\\
        - Hui et al.
    \item Summarize papers in writeup
    \item Create presentation slide for papers
\end{itemize}

\subsection*{3/31}
\begin{itemize}
    \item Work on presentation: general MHD, modes, motivation, etc.
    \item Continue editing writeup
\end{itemize}

\subsection*{4/07}
\begin{itemize}
    \item Add research to writeup and complete it
    \item Add research to presentation and complete it
\end{itemize}

\subsection*{4/14}
\begin{itemize}
    \item Discuess Revisions for paper and presentation
    \item Revise writeup
    \item Revise presentation
    \item Practice presentation
\end{itemize}

\subsection*{4/21}
\begin{itemize}
    \item Prepare for presentation on Monday
\end{itemize}


\end{document}
