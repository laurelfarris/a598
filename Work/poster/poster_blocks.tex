\documentclass{beamer}
\usepackage{verbatim} % For using /begin{comment}; /end{comment}
\usepackage{multirow}
\usepackage{booktabs}
\usepackage{textpos}
\usepackage{graphicx}
\usepackage{lmodern} % Font style
\usepackage{ragged2e} % For using \justify

\setbeamertemplate{frametitle}[default][center]
\addtobeamertemplate{block begin}{\vspace{-8pt}}{}

\setbeamerfont{frametitle}{series=\bfseries} % Frame titles should be bold

\definecolor{myred}{HTML}{AB4642}
\definecolor{myorange}{HTML}{DC9656}
\definecolor{myyellow}{HTML}{F7CA88}
\definecolor{mygreen}{HTML}{A1B56C}
\definecolor{myteal}{HTML}{86C1B9}
\definecolor{myblue}{HTML}{7CAFC2}
\definecolor{mypurple}{HTML}{BA8BAF}
\definecolor{mybrown}{HTML}{A16946}

\setbeamercolor{background canvas}{bg=white}
\setbeamercolor{normal text}{fg=black}
\setbeamercolor{frametitle}{fg=white}
%\setbeamercolor{framesubtitle}{fg=gsa}
%\setbeamercolor{block title}{fg=cblue}

\begin{document}

\setbeamercolor{frametitle}{bg=myred}
\begin{frame}{Abstract}
\justify{Coronal seismology involves the investigation of magnetohydrodynamic
(MHD) waves and oscillatory phenomena that arise in the solar corona.
Properties of the observed modes are largely dependent on their
environment, and therefore can be used to extract atmospheric
parameters that are otherwise difficult to observe.
The general theory behind MHD phenomena is investigated here, along with
the characteristics of different modes
and the information that can be extracted from them.
A few methods are applied to data from the \emph{Atmospheric Imaging
Assembly} (AIA) instrument on the \emph{Solar Dynamics Observatory} (SDO).}
\end{frame}
\setbeamercolor{frametitle}{bg=myblue}
\begin{frame}{Introduction}
\end{frame}
\setbeamercolor{frametitle}{bg=mygreen}
\begin{frame}{MHD Theory}
\end{frame}
\setbeamercolor{frametitle}{bg=myyellow}
\begin{frame}{Data}
\end{frame}
\setbeamercolor{frametitle}{bg=myorange}
\begin{frame}{Results}
\end{frame}
\setbeamercolor{frametitle}{bg=mypurple}
\begin{frame}{Conclusions}
\end{frame}
\setbeamercolor{frametitle}{bg=myteal}
\begin{frame}{Future Work}
\end{frame}

\end{document}
