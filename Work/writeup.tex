\documentclass[preprint2]{aastex}
\usepackage{natbib}
\usepackage{color}
\usepackage{booktabs}
\bibliographystyle{apj}

%\tabletypesize{\small}
%\rotate
%\tablewidth{dimen} % Default - pagewidth
%\tablenum{1}
%\tablecolumns{num}
%\tablecaption{Characteristic properties of the basic MHD waves,
%    along with their observational techniques\label{prop}}
%\tablehead{\colhead{} & \colhead{} & \colhead{}}

\shorttitle{Coronal Seismology}
\shortauthors{Farris}

\begin{document}

\title{\vspace{-0.75in}Coronal Seismology}
\author{\vspace{-0.25in}Laurel Farris}
\affil{New Mexico State University}
\email{laurel07@nmsu.edu}

\begin{abstract}
Coronal seismology involves the investigation of magnetohydrodynamic
(MHD) waves and oscillatory phenomena that arise in the solar corona.
Properties of the observed modes are largely dependent on their
environment, and therefore can be used to extract atmospheric
parameters that are otherwise difficult to observe.
The general theory behind MHD phenomena is investigated here, along with
the characteristics of different modes
and the information that can be extracted from them.
A few methods are applied to data from the \emph{Atmospheric Imaging
Assembly} (AIA) instrument on the \emph{Solar Dynamics Observatory} (SDO).
\end{abstract}
\keywords{Sun: corona{-}Sun: oscillations{-}Sun: seismology}

\section{Introduction}\label{intro}
The solar corona is characterized by various types of structures, such
as coronal loops, filaments, and bright points. These are formed by
plasma flowing along magnetic field lines and emitting light primarily in the
EUV\@. Disturbances on the sun
can be seen in the form of oscillations or waves propagating through such
structures, which act as waveguides.
Such disturbances can be triggered by flares and
coronal mass ejections (CMEs), or by continuous excitations from
photospheric motions and global seismological modes, such as pressure modes,
in the solar interior.
The behavior of these waves can be described using the theory of
magnetohydrodynamics (MHD). Different MHD modes have different characteristic
speeds, which are related to other physical properites of the corona, such
as magnetic field strength and density, which are
difficult to measure directly (\cite{tor_2}).

Here, the properties of MHD modes are investigated, along with literature
from a variety of authors in the field.
The basic concepts of MHD waves and oscillations are described in
\S\ref{MHD}, and
\S\ref{topics} covers several MHD modes in detail.
A description of a research project and the results are presented in
\S\ref{project}.
Conclusions and future work are discussed in
\S\ref{conclusion}.

\section{Basic MHD}\label{MHD}
\subsection{Types of Modes}
The relationship between MHD modes can be best understood
by their relative phase speeds. A dispersion diagram is shown in
figure {\ref{speeds}}, which plots the variation of phase velocity as a
function of wavenumber for different modes.
to the internal and external Alfv\'en and sound speeds.
Each of these curves represents a solution to the dispersion relation,
which is outside the scope of this paper.
\begin{figure}[!htb]
    \plotone{disp_diagram.png}
    \caption{Dispersion diagram showing the variation in phase velocity
        as a function of wavenumber. The phase velocity is expressed as a
        factor of the internal sound speed, and the wavenumber is
        multiplied by the factor $a$ to give it dimensionless units.
        Image credit:~\cite{Nak}}
    \label{speeds}
\end{figure}
There are two primary characteristic speeds determined by the properties
of the surrouding medium.
The acoustic, or sound speed ($C_{s}$), is driven by thermal pressure of
the gas, and is given by:
\begin{equation}\label{sound_speed}
    C_s = \sqrt{\frac{\gamma{P}}{\rho}}
\end{equation}
In contrast, the Alfv\'en speed is magnetically driven, and is
dependent upon the magnetic field strength.
It is given by:
\begin{equation}\label{Alfven_speed}
    C_A = \frac{B}{\sqrt{4\pi\rho}}
\end{equation}
In general, MHD waves are divided into two categories:
Alfv\'en waves and magnetoacoustic waves.
Magnetoacoustic waves can be \emph{slow} or \emph{fast}, depending on
how the velocity of the wave compares to the local sound speed,
$C_{s}$.
Slow-mode magnetoacoustic waves have phase speeds roughly equal to $C_{s}$
in the medium in which they reside, analogous to typical acoustic waves,
or sound waves.
Fast-mode magnetoacoustic waves have phase speeds comparable to the Alfv\'en
speed, or $\omega/k \approx V_A$ (\cite{kink_1}).

Each MHD mode is characterized based on a number of qualities, such
as its wavelength, period, lifetime, speed, and
whether it is propagating or standing, fast or slow, longitudinal
or transverse, etc. The driving mechanisms that give rise to each mode
can differ depending on these qualities and the environment in which
the modes reside, and is one of the important MHD topics under investigation,
along with the damping mechanism (\cite{kink_1}).

\subsection{Equations and Models}
MHD waves are often modeled with a straight, cylindrical flux tube embedded
in a uniform magnetic field. The interior values of magnetic field strength
($\mathbf{B}$), mass density ($\rho$), pressure ($P$), and temperature
($T$) are different from the exterior values.
%--------------------------------------------------------------------%
\subsection{Excitation and Damping Mechanisms}
\subsubsection{Resonant Absorption}
If the frequency of a wave falls within one of two types of MHD
continua (namely, the Alfv\'en or cusp continua), a resonance will
occur.
In the cylinder model, this location is a shell at radius $r$.
In the Alfv\'en continuum, the Alfv\'en wave is the mode that is
resonantly excited and dissipates in the corona.
This is one of the processes thought to contribute to the heating
of the corona. It involves the conversion from a collective mode
to a local mode, at a timescale much shorter than that
of the dissipative damping.
To analyze this mathematically, equations of ideal MHD cannot be used
because dissipation effects must be taken into account
(\cite{Nak}).
\subsubsection{Phase Mixing}
Another effect associated with wave modes from MHD continua is called
phase mixing, and is also a possible mechanism driving the heating of
the solar corona.
%--------------------------------------------------------------------%
\subsection{Observation Techniques}
Due to the lack of spatial resolution for the scales at which MHD
oscillations occur,
many are detected using time series analysis, where
the periodic changes in intensity are analyzed as a function of time.
As sufficient spatial resolution was developed, more observations
of physical phenomena were made possible,
such as the transverse displacement of coronal
loops by kink oscillations (see \S{} {\ref{kink}}).

Density variations are usually observed as intensity variations,
180$^{\circ}$ out of phase.
Oscillations are often observed in coronal loops as modes reflected back
and forth between the loop footpoints, which are
anchored in the photosphere and act as the wave nodes.
For insufficient spatial resolution (or line-of-sight (LOS) angle),
Doppler shifts can be extracted.

Often, it is the period of oscillation that is directly
measured from observations. The general relationship between period
and other wave parameters for the fundamental oscillating mode is
given by:
\begin{equation}
    P = \frac{\lambda}{C_{ph}}
\end{equation}
where $\lambda$ is the wavelength. For example, in a coronal loop of length
$\ell$, the wavelength is given by $\lambda = 2\ell$.

\section{Types of MHD Modes}\label{topics}
The behavior of MHD modes can vary based on whether they are fast or slow,
stading or propagating, etc. These are discussed below, along with short
literature reviews of work that has been done to observe, identify, and
utilize these modes to extract physical coronal parameters.

\subsection{Kink Oscillations}\label{kink}
Kink oscillations are directly observed in coronal loops in extreme
ultraviolet (EUV) wavelengths.
They characterize the spacial oscillations that occur over the surface of
the loop, perpendicular to the direction traversed by the length
of the loop (\cite{Nak}).
Kink oscillations generally are not accompanied by intensity variations;
they displace the axis of the magnetic structure in which they propagate,
but the cross-section of the waveguide remains roughly the same.
Kink oscillations occur in the ``long-wavelength regime'',
which is defined by
the product of the wavenumber and the cross-section of the coronal
loop being much less than 1 ($ka \ll 1$). In other words, the
wavelength of the oscillation is much greater than the
cross-section of the waveguide.
The phase speed is just above the Alfv\'en speed within the loop,
and the period of the oscillation is expected to be between
$\sim$ 2 and 20 minutes (\cite{Asc}).
Observations of kink oscillations are important to solar physics as
the magnetic field strength can be derived from the period.

Due to a lack of sufficient instrumentation,
spatial variations such as those caused by kink oscillations
were not resolved until the launch of the \emph{Transition Region And
Coronal Explorer} (TRACE) spacecraft.
Some of the first results from preliminary TRACE data were produced by
\cite{kink_1} after investigating oscillations detected in coronal loops.
Images in the 171 \AA{} and 195 \AA{} extreme ultraviolet (EUV) bandpasses
revealed several coronal loops undergoing transverse oscillations
triggered by a nearby class M4.6 solar flare that occurred in July of 1998.

To identify the type of mode, they first isolated the oscillating
loop by creating \emph{running difference} images to subtract out
the static background features. They then measured the period
of the oscillation by tracking the spatial displacement of the
isolated loop through time. The use of different locations along the
length of the loop was a significant diagnosis technique, as it
allowed for the discrimination between standing oscillations
and propagating waves, as long as at least half of the loop length
was detectable for determining any evidence of a phase shift.

Aside from the detection and identification of the kink mode,
these observations also explored the possible triggers of MHD
oscillations, a phenomenon that is not yet well constrained. The
similarity in period and phase of all five loops indicated a common
triggering mechanism, most likely the nearby flare. As there was some
physical distance between the flare site and the loops, the excitation
must have been a disturbance generated by the flare that then
traveled to the footpoints of the loops.

The observations had enough qualities characteristic of
fast kink modes, including the spatial displacements characteristic
of this asymmetrical mode, to identify these oscillations as kink modes.
An average period of $T = 269 \pm 6$ seconds, or
roughly 4.5 minutes, was obtained, which fit well with the theoretical
period for kink oscillations.
The absence of any phase
shift along the length of the loops revealed that these were
\emph{standing waves}, with nodes located at the loop footpoints.

The ability to identify kink modes has great significance
for solar physics due to a correlation between the oscillation period,
$P$, and the magnetic field strength, $B$, of the loop:
\begin{equation}
    P \propto \frac{L}{\sqrt{n}}{B}
\end{equation}
where $L$ is the loop length and $n$ is the number density,
both of which can be (roughly) estimated from known coronal conditions.
Kink waves are \emph{Alfv\'enic}; their phase speed is
close to that of the internal Alfv\'en speed:
$C_{k} \gtrsim V_{A}   $

More recently, \cite{kink_2} investigated the driving mechanism
behind the production and damping of the kink mode.
This was found to occur on rapid timescales for both standing
and propagating kink modes, both of which are consistently found
throughout the solar corona.
However, the mechanism behind the rapid damping of these modes was
unclear.
The gravitational stratification of the atmosphere could have
the effect in increasing the velocity amplitude, thus
reducing the attenuation caused by mode-coupling.

They compared two possible functional forms of the damping profile
of the driver: that of a Gaussian and an exponential form.
The Gaussian was a potential form of the amplitude variation at a
lower height ($z$), while the variation at higher $z$ took the
form of an exponential.
While the noise level of the data was too high to distinguish
between the two forms, the simulations followed the form of a
Gaussian.

They also considered the effect of the spatial profile of the driver
itself on the excitation and subsequent damping of the kink
waves. Two different possibilities were explored here:
the effect of a ``highly structured'' driver including only the
exponential damping profile, which they
found to be unrealistic, and
the effects of small-scale (i.e.\ less than the size of the loop cross-section)
of eddies and photospheric motions around the footpoints of the loops.
These motions can excite kink modes along the length of the coronal
loop, but not as efficiently as those produced by larger scale
motions. These motions are what determine the properites of the
observed modes.
The observed damping of the kink mode was found to be a result
of mode-coupling and subsequent energy transfer from the characteristic
transverse motions of the kink mode, to azimuthal motions.
The period observed by \emph{CoMP} was centered around
5 minutes, fitting within the expected kink period of 2-20 minutes.
The phase speed was around 0.6 Mm s$^{-1}$, leading to a calculated
wavelength of about 180 Mm.

\subsection{Sausage Oscillations}\label{sausage}
In contrast to kink oscillations, sausage oscillations do not displace
the axis of the structure in which they reside. However, they do cause
a periodic change in the cross-section of the waveguide, and hence
are observable through changes in intensity (and therefore density,
due to flux conservation). Sausage oscillations only exist shortward
of the long wavelength cutoff, where the wavelength is comparable to
the cross-section of the loop in which they oscillate. Therefore
they can only exist in loops that are ``sufficiently thick and dense''
(\cite{pfw_2}).

\cite{sausage_1} plotted the changes in intensity and cross-sectional
area for sausage oscillations in photospheric pores extending up
through the solar atmosphere. They used the general cylinder model for
the pores, though it is more likely that the cross-sectional area of
the waveguide increases with height as the temperature increases and
density decreases.
Sausage waves are characterized by a change in the cross-section,
but no displacement of the loop axis. They occur on much shorter
timescales than kink waves.
The relationship between pore size and intensity can
indicate

\cite{sausage_2} investigated oscillations associated with magnetic
pores, which are essentially small sunspots with an upper limit of
about 1700 km in diameter and consist of an umbra, but no penumbra.
These features have high magnetic strength
($\sim$ 1700 G) in the photosphere and expand in diameter as the
height above the
photosphere increases into the chromosphere. As magnetic flux is
always conserved, observing the size of these features in the upper
layers of the atmosphere can reveal their size in the photosphere,
which is not as easily observable. However, this requires knowledge
of their magnetic strength in the atmosphere as well.
These pores act as a waveguide (much like the way coronal loops act
as a waveguide). Observations along the line of sight reveal periodic
changes in intensity due to the change in cross-section characteristic
to sausage modes. These two variations are 180$^{\circ}$ out of phase,
as an increase of surface area corresponds to a decrease in density,
and hence, a decrease in intensity. The observed period for These modes
was about 30-450 seconds, and they were considered as a
possible heating mechanism for the corona. However, some wavelengths are
reflected at the transition regions, while other make it through,
and the ones that do make it through are not abundant enough to heat
the corona to the temperatures that are observed.
%-------------------------------------------------------------------%
\subsection{Acoustic Oscillations}

\cite{acoustic_2} used data from the
\emph{Extreme Ultraviolet (EUV) Imaging Spectrometer} (EIS) on board the
\emph{HINODE} spacecraft to derive periodicities of disturbances
observed above bright points (BPs) in the corona.
While the propagation of magnetoacoustic waves can be observed
in the solar atmosphere, they do not carry enough energy to physically
propagate there (and thus were ruled out as a possible mechanism of
heating the corona or accelerating the solar wind).

They used intensity oscillations of a few different ionization
species to pinpoint the origin and progression of magnetoacoustic waves
between bright points in the photosphere, through the transition regions,
and up into the corona. The time series of intensity oscillations was
converted into a power spectrum, and periodicities were extracted using
wavelet analysis and periodograms (note here on what exactly these are?).
The periods they derived
($\sim 263 \pm 80$ s for the He II 256.32 \AA{} emission line and
$\sim 241 \pm 60$ s for the Fe XII 195.12 \AA{}  emission line)
were close to the 5-minute global oscillations of the sun.

The relatives periodicities of acoustic oscillations were recognized
as important characteristics, and modeled by~\cite{acoustic_1}.
They recognized magnetoacoustic oscillations as a useful way to determine
other properties of the solar atmosphere.
%-------------------------------------------------------------------%
\subsection{Propagating Acoustic Waves}
% Propagating vs. standing
The MHD modes that have been discussed up to this point are considered
\emph{standing} modes: oscillations with nodes in a fixed
position, such as the footpoints of coronal loops anchored in
the photosphere, or the points at the photosphere and chromosphere where
acoustic oscillations with frequencies less than the cutoff are trapped.
In contrast, \emph{propagating} waves have nodes that move with time
(though two propagating waves traveling in opposite directions can
have a net result of a single, standing oscillation; see~\cite{Asc},
ch. 8).
Propagating acoustic waves have been observed in both closed loops
(where the wave decays before it can reach the other footpoint
and reflect back; see {\cite{acoustic_1}})
and open structures.
Waves with propagations speed much lower than the
local Alfv\'en speed, are categorized as slow magnetoacoustic waves.
These waves travel along magnetic field lines at speeds roughly equal to
the local sound speed.

The changes in intensity along the same location as these waves
propagate in time are mapped side by side to give time-distance
information. The period, wavelength, propagation speed, and amplitude
can all be derived using this method.
Since the local sound, $C_{s}$ is related to temperature, $T$, as
\begin{equation}\label{sound-temp}
 c_{s} \propto \sqrt{T}
\end{equation}
a difference in observed propagation speeds implies a changing
temperature profile in the transverse direction across the loop
(\cite{Nak}).

Propagating acoustic waves are generally observed as ``propagating
disturbances'' awaiting further diagnoses of magnetoacoustic waves,
whose amplitudes are rather weak, requiring careful, detailed data
analysis to acquire good signal-to-noise (S/N).

One of the first studies to analyze
simultaneous observations at different wavelengths was carried out by
\cite{pac_1} using data from two different instruments:
the Extreme ultraviolet Imaging Telescope (EIS)
on the Solar and Heliospheric Observatory (SOHO) and the
Transition Region And Coronal Explorer (TRACE).
These wavelengths, along with their corresponding ions emitting at those
wavelengths, the temperature, and other relevant quantities from the
study are given in table {\ref{stuff}}.
%-------------------------------------%
\begin{table}[h]
\centering
\begin{tabular}{c c c}
\hline\hline
instrument & EIT & TRACE\\
\hline
ion & Fe {\footnotesize XII} & Fe {\footnotesize IX}\\
wavelength & 195 \AA{} & 171 \AA{}\\
cadence & 15 s & 25 s\\
temperature & 1.6 MK & 1 MK\\
sound speed & 192 km s$^{-1}$ & 152 km s$^{-1}$\\
propagation speed & 110 km s$^{-1}$ & 95 km s$^{-1}$\\
\hline\hline
\end{tabular}
\caption{Relevant quantities from \cite{pac_1}}
\label{stuff}
\end{table}
%-------------------------------------%
Both instruments observed AR 8218, where the presence of
``weak transient disturbances'' were revealed, and later classified as
slow, propagating magnetoacoustic waves, with speeds that varied
between 65 and 150 km s$^{-1}$.
The expression for the formal sound speed of the ambient region
was given by
\begin{equation}
    c_{s} = 152\ \sqrt{T}\ \textrm{m}\ \textrm{s}^{-1}
\end{equation}
where $T$ is in units of Kelvin.
This was compared to the observed speed of the propagating wave, which
was derived from the slope of the ``ridges'' on each of the four
time-distance plots.
The propagation speed of each wave was slightly ($\sim$ same order of
magnitude) lower than the local sound speed. This difference
was due to the angle between the loop and the plane of the sky against
which the observations were made.
The amount of time for a disturbance to pass a particular point was
determined to be $\sim$ 169 s from the TRACE data.
As both observations targeted the same wave, the significance of the
differing speeds for each observation indicated a temperature gradient
in the loop itself, indicating either a bundle of loop threads that
make up a single loop, or a number of concentric shells that make up a
single loop.

Because of the low amplitude of slow magnetoacoustic waves, their
speeds can be difficult to determine observationally. Many techniques
have been developed to extract these speeds with higher signal-to-noise
ratio. One such technique
involves the resonant behavior of ``surfing signals''
revealed by quasi-periodic disturbances (\cite{pac_2})
The method did not rely on a well-defined periodicity, which made it
particularly useful for slow magnetoacoustic waves, as they tend to
exhibit quasi-periodic characteristics. It also had the capability
to extract both negative and positive velocities, i.e.\ waves traveling
both up and down coronal loops.

Since the phase speed of slow magnetoacoustic waves is determined by
the local sound speed where they are produced ($\propto\sqrt{T}$),
the wave speeds along
different loops with different temperatures were expected to vary,
but the frequencies were expected to be the same for waves excited
by the same mechanism.

To test their technique on this temperature dependence for propagation
velocities, {\cite{pac_2}} used
about 6 hours of observational data from the
Atmospheric Imaging Assembly (AIA) on the Solar Dynamics Observatory (SDO).
The technique was applied to the lines at
131 \AA{}, 171 \AA{}, 193 \AA{}, and 211 \AA{},
over the active region (AR) NOAA AR 11082, which was free of sunspots
and high-energy flares.
Though the actual driving mechanism behind the observed disturbances
was uncertain, they did find
the same relationship between sound speed and temperature
as expressed in equation {\ref{sound-temp}}.
The observed propagations were
travelling primarily in the upward direction away from the footpoint,
at speeds of about 40 to 180 km s$^{-1}$ and periodicities between 4
and 8 minutes over all four channels.

The lack of dependence on linearity or consistent periods, this could
be useful method for analysis of mixed modes or damping waves.
Additionally, this study confirmed the temperature-dependence of
slow-mode velocities in non-flaring locations free of sunspots.
%Each surfing signal
%(ST) is a periodic disturbance as a function of time and frequency
\subsection{Propagating Fast Waves}%------------------------------------%
% General stuff about Moreton waves; where they are, what triggers them,
% characteristic speeds, significance of H-alpha, the search for coronal
% counterparts...
MHD modes that are categorized as ``fast'' (magnetoacoustic?) propagate
at $\sim$ 570-800 km s$^{-1}$. These types of waves were investigated by
\cite{pfw_1} using observations of a type of fast propagating wave called a
\emph{Moreton}. These waves are
typically triggered by energetic flares and CMEs, with lifetimes of
about 10 minutes.
They were described as the intersection of MHD fast waves shock-propagating
between the chromosphere and corona, and are observed in H$\alpha$.
While they are rarely observed, their
counterparts can be seen in the corona in the form of EIT waves
(named after the \emph{EUV Imaging Telescope} on the
\emph{Solar and Heliospheric Observatory}).

\cite{pfw_1} presented the first simultaneous observations of both
a Moreton wave in H$\alpha$ images, obtained from the
\emph{Solar Magnetic Activity Research Telescope} (SMART),
along with 193 \AA{} images from AIA
(Fe{\footnotesize XII}, at a temperature $T$=10$^{6.1}$ K),
and 195 \AA{} and 304 \AA{} images from the
\emph{STEREO} instrument.
The so-called {EIT} waves are EUV waves
named after the instrument that first detected them,
the \emph{EUV Imaging Telescope} (EIT) on the
\emph{Solar and Heliospheric Observatory} (SOHO).

% Analysis
The Moreton wave was triggered by
a class-X6.9 flare from AR NOAA 11263 that occurred on August 9, 2011.
It lasted for $\sim$ 6 minutes, with an average
velocity of $\left\langle v \right\rangle$ = 760 km s$^{-1}$.
The EUV waves were thought to be the coronal counterparts of the
Moreton wave, although the difference in velocity and propagation
direction showed that this was not the case.
The EIT wave propagated at a velocity of 200-400 km s$^{-1}$ and lasted for
about 45-60 minutes, while the Moreton wave propagated at a velocity
of 500-1500 km s$^{-1}$ and only lasted for 10 minutes.
Moreover, Moreton waves are quasi-isotropic, while the EIT waves were observed to
propagate isotropically.

Also observed were a filament and a promiance (two separate structures)
that were near the region where the flare ocurred. The waves that resulted
from the flare passed by these two structures, causing them to oscillate,
and thus provide a valuable observable for diagnoising the propgating
disturbances.
The velocity of the propagating coronal wave was derived to be
$\sim$ 800 km s$^{-1}$ from observations of the prominance, and
$\sim$ 570 km s$^{-1}$ from observations of the filament.
These numbers provided the expected velocity range for the coronal
waves, placing them in the category of fast-mode MHD waves.

\begin{table}[h]
    \centering
    \begin{tabular}{c c}
         \hline\hline
         Wave & velocity [km s$^{-1}$]\\
         \hline
         F1 & 760\\
         F2b & 730\\
         F2f & 620\\
         S2b & ?\\
         F3b & 550\\
         S3b & 340\\
         F3f & 580\\
         \hline
    \end{tabular}\\
    \caption{Waves observed by~\cite{pfw_1}}
    \label{table:lines}
\end{table}

% Add movie somewhere in here!

Another investigation of propagating fast waves was carried out by
\cite{pfw_2}.
They observed intensity perturbations of ``quasi-periodic propagating fast
magnetoacoustic wave trains'' associated with AR 11227.
These wave trains appeared $\sim$ 110 Mm from the center of the flare,
2.2 minutes after the flare occurred.
They were only visible in the AIA 171 \AA{} bandpass, suggesting a
location restricted to a certain height range above the photosphere.
They derived an average velocity of
833 km s$^{-1}$, which is in the Alfv\'en range.
Three wavetrains were observed, travelling one behind the other
with different wavelengths (periods) and different (initial) phase
speeds (although they all ended with a phase speed of about 600 km s$^{-1}$.
Comparisons between these waves and the radio spectrum of the flare
acquired from the \emph{Nan\c{c}ay Radioheliograph (NRH)}
revealed that each wave train was triggered by a burst of radio energy
caused by the acceleration of non-thermal electrons above the
magnetic reconnection site. The periods of such waves follow a
characteristic ``tadpole'' pattern, where the period decreases at
a fixed height, which is 150 Mm here.

These could possibly be the kink mode if the waveguide was in
the form of a loop. These can only be identified if the observed loop
segment has the correct line of sight with respect to the observer.
A longer wavelength results in a faster speed, which then results
in the dispersion of fast magnetoacoustic waves. The range in
propagating phase speeds they derived was
$ v_{ph} = 735 - 845 $ km s$^{-1}$, with a final speed
(post-dispersion) of 600 km s$^{-1}$.

\subsection{Torsional Modes}
Torsional modes, or more commonly, \emph{Alfv\'en} modes, are axisymmetric
modes whose speed is determined primarily by magnetic pressure forces
(see equation~\ref{Alfven_speed}).
They are transverse waves that are highly anisotropic (\cite{Goossens}),
with a \emph{group} speed that
travels strictly along magnetic field lines in response
to a restoring force that acts to resist shear.
The \emph{phase} speed can be at an angle $\theta$ to the direction
of the magnetic field $\vec{B}$:
\begin{equation}
    V_A = \pm \frac{B}{\sqrt{4\pi\rho}}\cos\theta
\end{equation}
The energy carried by this wave is contained in the group speed
(\cite{Somov}).
%$\ldots$ this source also has a good phase velocity diagram).

In the classic cylindrical model of a plasma structure,
Alfv\'en waves undergo torsion (aka, twisting), in opposite directions on
either end of the cylinder (\cite{Nak}).

The primary damping mechanism for Aflv\'en waves is resonant absorption,
and is thought to be a possible cause for coronal heating.
However, they are difficult to observe because, like the kink mode,
they do not cause an intensity change, and additionally do not displace
the structure through which they travel, so there is no visual indication
that they are present.

\cite{tor_1} derived the magnetic field strength and Alfv\'en velocity
using coronal seismology, and compared them to other techniques to
confirm that coronal seismology was in fact a reliable method to determine
these parameters. For the direct techniques,
the magnetic field was found using magnetic extrapolation,
and the Alfv\'en velocity was found
using spectroscopic observations of the Fe
{\footnotesize XII} $\lambda$ 195.12 line, where the Doppler shifts
along the line-of-sight (LOS) in a coronal loop could be used to
measure velocities. For the seismological applications, the AIA
171 \AA{} data was used.

They also took into account the situation of a non-zero density
gradient and magnetic stratification, a more realistic consideration
than the ideal case of a flux tube with constant density inside
and zero density outside (the contrast between the two is one of
the aforementioned difficult parameters to determine in the corona).

The phase speed of the loop oscillation was found to be between
2500 and 2600 km s$^{-1}$. This range fit the characteristic kink
speed, given by
\begin{equation}\label{kink_speed}
    C_k \approx \sqrt{\frac{2}{1+\zeta^{-1}}}V_A
\end{equation}
where $\zeta$ is the ratio of the internal loop density to the external
density, $\rho_i/\rho_e$.

The results allowed for the conclusion that coronal seismology is a
reliable method for deriving these coronal parameters.

They did note that oscillations do not necessarily indicate MHD waves,
as these same oscillations can result from upflows of jets, which were
found to be the dominate source of the oscillations seen around the loop
footpoints. Toward the top of the loop, the observed oscillations were
in fact found to be due to either kink or Alfv\'en waves
They also noted that intensity changes can occur as
a result of the loop itself begin in motion, transversing back and
forth across the slit location, and cautioned that a phase difference
between intensity and Doppler shift was not an immediate indication
of density perturbation, and further analysis should be carried out.

Alfv\'en waves can be difficult to observe (or at least difficult to
identify observed disturbances as Alfv\'en waves) since, like kink
oscillations, they do not cause any periodic changes in intensity or density.
It follows that determining the Alfv\'en speed is also difficult,
depending on what parameters are directly observable.

\cite{tor_2} tested the accuracy of the coronal seismology technique by
using those methods to determine the Alfv\'en speed, and hence the magentic
field strength, and comparing the
results to those from more ``direct'' methods of determining magnetic field
strength, namely, magnetic extrapolation and spectroscopy.
An X-class flare triggered oscillations in two coronal loops in AR
1123 in 2011.
The loops were observed in the AIA/SDO 171 \AA{} bandpass
as transverse oscillations with periods of
$\sim$ 2-3 minutes. Using the expression for the fundamental standing
mode ($P = 2\lambda/V_{ph}$), the observed period and loop length were
used to calculate the phase velocity, which was around 2500 km s$^{-1}$.
Due to the tranverse nature of the oscillations and the relatively short
periods, these were identified as kink oscillations. Using the expression
for the kink speed in the low plasma-$\beta$ environment, along with the
assumption of a uniform density and magnetic field along the length of
the loop, setting the phase velocity equal to the kink speed and solving
for the Alfv\'en speed gives
\begin{equation}
    V_{A} = V_{ph}\sqrt{\frac{1 + \rho_{e}/\rho_{i}}{2}}
\end{equation}
The ratio between the external ($\rho_{e}$) and internal ($\rho_{i}$)
densities is a difficult value to measure, and is one of the primary reasons
that Alfv\'en speeds are so hard to constrain.
Here they used the DEM inversion technique to derive the density based
on the observed intensity, and the magnetic field strength was extrapolated
from the three-dimensional model of the coronal loops, represented as
potential fields.
They concluded that the two methods produced results similar enough
to consider coronal seismology a reliable method. Previous studies
had produced magnetic field strengths up to three times higher than
those calculated using coronal seismology, but here this was thought
to be due to a lack of consideration of loop stratification,
rather than assuming constant density and magnetic field strength
along the length of the loop, which extends far enough into the
corona to undergo an overall gravitational stratification.

\section{Discussion}\label{disc}
The observational techniques and relevant properties of each of the different
kinds of MHD waves are summarized in table 1.
%----table----%
\begin{table*}[ht]
    \centering
    \begin{tabular}{c c c c}
        \hline\hline
        Type of wave &
        Timescales &
        Sizescales &
        Observational techniques\\
        \hline
        Kink Oscillations & short & short & something\\
        Sausage Oscillations & short & short & something\\
        Acoustic Oscillations & x & x & x\\
        Propagating & x & x & x\\
        \hline\hline
    \end{tabular}
\end{table*}
%-------------%

\section{Research Project}\label{project}
As part of the general topic of coronal seismology,
a small research project was carried out.
Several of the observational
and analytical methods employed in the literature were emulated for
this project, in addition to a cross-correlation analysis.
\subsection{Data}
An hour of data from AIA/\emph{SDO} at 193 \AA{} was examined for possible
wave activity or notable disturbances. Each image in the time series was
separated by 12 seconds, and taken during July of 2012. The first image in
the series is shown in figure {\ref{full}}.
\begin{figure}[!htb]
    \plotone{full_disk.png}
    \caption{First image in the time series from AIA/\emph{SDO}.}
    \label{full}
\end{figure}
The image clearly shows an active region in the lower right region
and a coronal hole in the upper left, with areas of quiet sun in between.


\subsection{Lightcurves}
To determine if there was a possible relationship between the size of the
bright point and the intensity, the time variation of both quantities were
plotted as shown in figure {\ref{lc}}.
\begin{figure*}[!htb]
    \plotone{plot2.png}
    \caption{Size and intensity (counts per pixel) of the bright point as
        functions of time throughout the entire data series.}
    \label{lc}
\end{figure*}
There appears to be a possible correlation between the two, but further
analysis is required, such as a fourier transform.

\subsection{Cross-correlation}
To check for variation or disturbances in the horizontal direction, i.e.\
outward from the center of the bright point in the radial direction, a
cross-correlation was run between a single pixel located roughly in the
center of the bright point, and every other pixel in the image shown in
figure {\ref{}}.
\begin{figure}[!htb]
    \plotone{bp1_contour.png}
    \caption{Image of the single bright point analyzed for this research.
        The contours outline the estimated size of the bright point
        using an intensity cutoff. The areas in the upper region of the
        image were also above the intensity threshold used to determine
        the size of the bright point, and hence are also outlined. The circle
        in the center indicates the location of the pixel that was correlated
        with every other pixel in the image.}
    \label{tt_color}
\end{figure}


\begin{figure*}[!htb]
    \plottwo{cc_color.png}{tt_color.png}
    \caption{{\sc Fig.~2} (\emph{a}) Image illustrating the maximum correlation
        value of each pixel with the central pixel;
        (\emph{b}) Image of the timelag corresponding to the maximum correlation
        value of each pixel in \emph{a}.
       \label{cc_tt}}
\end{figure*}

\section{Conclusion}\label{conclusion}
The data set examined here requires further analysis, such as applying a
Fourier Transform and finding exact locations for high correlation values.
Data from the \emph{Helioseismic and Magnetic Imager} (HMI) on \emph{SDO} 
might be worth examining to see if a counterpart to the bright points can
be seen in the photosphere.
These techniques could be applied to the active region in the same data set,
assuming they produce reliable results from the ``cleaner'' data in the
coronal hole.
\newpage
\bibliography{reffile}
\end{document}
