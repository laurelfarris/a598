\documentclass[12pt]{article}
\usepackage[margin=1in]{geometry}
\usepackage{titlesec}
\usepackage{mdwlist}
\setlength{\parindent}{0em}
\setlength{\parskip}{0.5em}
\usepackage{enumitem}
\usepackage{hyperref}
%\setlist[1]{itemsep=-2pt}

\titleformat*{\section}{\Large\bfseries}
\titleformat*{\subsection}{\large\bfseries}

\title{\vspace{-0.5in}Sun Stuff}
\author{}
\date{}

\begin{document}
\maketitle

\vspace{-1in}

\href{http://solarscience.msfc.nasa.gov/feature3.shtml}
{textcolor{blue}{link!}}

\section*{A}

\subsection*{active regions}

\subsection*{Alfv\'en waves}

\section*{B}
\subsection*{Bright points}

\section*{C}

\subsection*{coronal holes}

\subsection*{coronal loops}
\begin{itemize*}
    \item \emph{Modelled} as flux tubes; probably consist of
        many flux tubes.
\end{itemize*}

\subsection*{coronal mass ejections (CMEs)}
Release of magnetic energy; reach Earth in a few days.

\subsection*{faculae}
Aka.\ ``little torch''
    \begin{itemize*}
        \item Appear in \emph{photosphere}; same thing as plage
            (which appear in the chromosphere).
        \item Bright spots --- reason why total brightness is higher at
            solar maximum.
        \item small scale bright points in the vicinity of sunspots;
            appear hours before the sunspots, but can remain for months
            after the sunspots are gone.
        \item visible only near limb.
    \end{itemize*}

\subsection*{filament}
    \begin{itemize*}
        \item Viewed on disk; same thing as prominences.
        \item Thin, cool, dark ribbons
    \end{itemize*}

\subsection*{flares}

\subsection*{flux tubes}
    \begin{itemize*}
        \item Formed deep in the convection zone.
        \item Rise by magnetic buoyancy in an $\Omega$-shaped loop.
        \item Magnetic field lines can be thought of as infinitely
          thin flux tubes.
    \end{itemize*}

\subsection*{Fraunhofer lines}
First observed in 1802 by William Hyde Wallaston, and carefully
studied by Joseph von Fraunhofer (1787--1826), who catalogued
the wavelengths in 1814. Ionized sodium: inert gas structure,
no lines in visible regime. Dark absorption features in solar
spectrum, in cooler layers of the atmosphere.
\begin{itemize*}
    \item H and K lines of ionized Calcium
    \item D line of neutral sodium (5890 \AA{}, 5896 \AA{};
        resonance lines between n=1 and n-2).
    \item E (iron)
    \item C and F (hydrogen)
\end{itemize*}

\subsection*{frozen-in flux}
In a perfectly conducting material (i.e.\ $\eta = 0$),
Ohm's law goes from
$ \vec{E} + \vec{v} \times \vec{B} = \vec{J}\eta $ to
$ \vec{E} + \vec{v} \times \vec{B} = 0 $
Nothing can be perpendicular to the field lines $\ldots$
See Alfv\'en's Theorem.

\section{J}
\subsection*{jets}
Word for rapid burst of emission? Rapid upflows, plasma
ejections$\ldots$ UV spectral emission requires high temperatures.
For UV:\ flux from photospheric continuum is low. Chromosphere:
temp is lower and background is higher (compared to $\ldots$?)
so lines are in absorption. Low flux: radiative excitation doesn't occur.
High temps allow for collisional excitation and emission upon returning
to the ground state. $\Delta v = \sqrt{v_{th}^2+v_{Nth}^2}$.

\section{K}
\section{L}
\section{M}
\section{N}
\section{O}

\section{P}
\subsection*{plage}
    \begin{itemize*}
        \item Appear in \emph{chromosphere}; same thing as faculae.
        \item Bright spots caused by light emitted by clouds of
            hydrogen or calcium (specifically H$\alpha$ and
            Ca H and K lines.
    \end{itemize*}

\subsection*{plume}
Apparently help to shield Earth from solar storms.
They are long thin streamers that project outward from the Sun's north and
south poles. We often find bright areas at the footpoints of these
features that are associated with small magnetic regions on the solar
surface. These structures are associated with the \emph{open} magnetic
field lines at the Sun's \emph{poles}. The plumes are formed by the action of
the solar wind in much the same way as the peaks on the helmet
streamers.

\subsection*{pores}
Small sunspots with an umbra, but no penumbra.
Size on order of upper limit of magnetic bright points ($\sim$ 1700
km).

\subsection*{prominence}
    \begin{itemize*}
        \item Viewed on the limb; same thing as filaments.
        \item May erupt sometime during its life and be associated
        with a CME
    \end{itemize*}

\section{Q}
\section{R}
\section{S}
\subsection*{solar wind}
Stream of energized, charged particles, primarily protons and electrons,
flowing outward from the sun at $v \leq 900$ km s$^{-1}$ and T = 10$^6$ K.
Solar wind plasma originates in thin, intense flux tubes at granule
and supergranule boundaries. Fast (steady) vs.\ slow (variable) wind.

\subsection*{spicules}
Dynamic jet of about 500 km diameter in the \emph{chromosphere} of the Sun.
It moves upward at about 20 km/s from the photosphere. Generated in the
chromosphere, where magnetic flux is concentrated, which outline boundaries
of supergranule network.

\subsection*{sunspots}
Dark regions of intense magnetic field.

\subsection*{supergranules}
Show up most clearly as a pattern of horizontal motions.
\begin{itemize*}
    \item Doppler measurements near limb
    \item local correlation tracking of granules near center
\end{itemize*}
Above a supergranule cell
\begin{itemize*}
    \item 750 km active regions
    \item 1600 km quiet regions
\end{itemize*}
Magnetic field spreads out to fill the chromosphere and form a
horizontal canopy or partial canopy.

\subsection*{surges}

\end{document}
