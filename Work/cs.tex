\documentclass{article}
\renewcommand\familydefault{\sfdefault}
\usepackage[margin=1.5in]{geometry}
\usepackage{graphicx}
\setlength{\parindent}{0em}
\setlength{\parskip}{0.5em}
\usepackage{enumitem}
\usepackage{xcolor}
\usepackage{amsmath}

\usepackage{setspace} % for toc spacing
\usepackage[toc]{multitoc}
%\setlength{\columnseprule}{0.5pt}
\usepackage{minitoc} % Doesn't appear to do anything

\usepackage{sectsty} % for customizing sections
\sectionfont{\color{black}}
\subsectionfont{\color{black}}
\usepackage{titlesec}
\titlespacing*{\section}{-0.5in}{2ex}{1ex}
\titlespacing*{\subsection}{-0.25in}{1ex}{0ex}

\setcounter{secnumdepth}{1}

\usepackage{enumitem}
\setlist{noitemsep, topsep=0ex}

\usepackage{hyperref}
\definecolor{darkpowderblue}{rgb}{0.0, 0.2, 0.6}
\hypersetup{colorlinks=true, linkcolor=black, urlcolor=darkpowderblue}
\urlstyle{same}

%----------------------------------------------------------------------------------
\begin{document}
\addtocontents{toc}{\protect\setstretch{0.1}}
\tableofcontents
\url{http://solarscience.msfc.nasa.gov/feature3.shtml}
\newpage
\hypersetup{linkcolor=darkpowderblue}


\section{Atmosphere}
\subsection{Chromosphere}
\subsection{Corona}
Strength and direction of magnetic field is hard to measure here.

\subsubsection{Emission lines}
\begin{itemize}
    \item Fe XIII 10747\AA{}
\end{itemize}

\section{Solar features}
\subsection{Granules}
\subsection{Supergranules}
Show up most clearly as a pattern of horizontal motions.
\begin{itemize}
    \item Doppler measurements near limb
    \item local correlation tracking of granules near center
\end{itemize}
Above a supergranule cell
\begin{itemize}
    \item 750 km active regions
    \item 1600 km quiet regions
\end{itemize}
Magnetic field spreads out to fill the chromosphere and form a
horizontal canopy or partial canopy.

\subsection{Flux tubes}
    \begin{itemize}
        \item Formed deep in the convection zone.
        \item Rise by magnetic buoyancy in an $\Omega$-shaped loop.
        \item Magnetic field lines can be thought of as infinitely
          thin flux tubes.
    \end{itemize}

\subsection{Frozen-in flux}
In a perfectly conducting material (i.e.\ $\eta = 0$),
Ohm's law goes from
$ \vec{E} + \vec{v} \times \vec{B} = \vec{J}\eta $ to
$ \vec{E} + \vec{v} \times \vec{B} = 0 $
Nothing can be perpendicular to the field lines $\ldots$
See Alfv\'en's Theorem.

A property of a moving fluid which represents the potential for helical flow
(i.e. flow which follows the pattern of a corkscrew) to evolve. Helicity is
proportional to the strength of the flow, the amount of vertical wind shear,
and the amount of turning in the flow (i.e. vorticity). Atmospheric helicity is
computed from the vertical wind profile in the lower part of the atmosphere
(usually from the surface up to 3 km), and is measured relative to storm
motion. Higher values of helicity (generally, around 150 m2/s2 or more) favor
the development of mid-level rotation (i.e. mesocyclones). Extreme values can
exceed 600 m2/s2.

\subsection{Bright points}

\subsection{Active regions}
Appear as bright \textit{plages} (\S \ref{plage})
of emission in the equatorial belt within $\pm$
30 degrees of the equator and represent moderate concentrations of magnetic flux
with mean field of 100 G or so.

\subsection{Pores}
Small sunspots with an umbra, but no penumbra. Size on order of upper limit of
magnetic bright points ($\sim$ 1700 km).

\subsection{Sunspots}
Dark regions of intense magnetic field.

\subsection{Spicules}
Towers of gas observed in the chromosphere, $\sim$ 500 km above the
photosphere. Last 5-10 minutes. Jets are about 500 km diameter and move upward
from the photosphere at speeds of about 20 km/s. They are regions of intense
energy, possibly caused by plasma being propelled by shock waves that
originated in the interior. Acoustic waves from photosphere shock and heat
plasma inside magnetic structures. Potential mass for solar wind and energy for
corona.

Two types of spicules:
\begin{enumerate}
    \item Type I spicules (dynamic fibers) are caused by shock waves.
        Fibrils driven by slow-mode magnetoacoustic shocks, sawtooth shape
        in chromospheric emission lines.
    \item Type II spicules are caused by strong flows. Rapidly evolving.
\end{enumerate}

\subsection{Facula}\label{ssec:facula}
Aka.\ ``little torch''
    \begin{itemize}
        \item Appear in \emph{photosphere}; same thing as plage
            (which appear in the chromosphere).
        \item Bright spots --- reason why total brightness is higher at
            solar maximum.
        \item small scale bright points in the vicinity of sunspots;
            appear hours before the sunspots, but can remain for months
            after the sunspots are gone.
        \item visible only near limb.
    \end{itemize}
\subsection{Plage}
    \begin{itemize}
        \item Appear in \emph{chromosphere}; same thing as \hyperref[ssec:facula]{faculae}
        \item Bright spots caused by light emitted by clouds of
            hydrogen or calcium (specifically H$\alpha$ and
            Ca H and K lines).
    \end{itemize}

\subsection{Prominence/Filament}
Prominence: Viewed on the limb against the dark sky;
may erupt sometime during its life and be associated with a CME.

Filament: Thin, cool, dark ribbons viewed against disk.

\subsection{Plume}
Apparently help to shield Earth from solar storms.
They are long thin streamers that project outward from the Sun's north and
south poles. We often find bright areas at the footpoints of these
features that are associated with small magnetic regions on the solar
surface. These structures are associated with the \emph{open} magnetic
field lines at the Sun's \emph{poles}. The plumes are formed by the action of
the solar wind in much the same way as the peaks on the helmet
streamers.

\subsection{Coronal streamer}
A wisp-like stream of particles traveling through the Sun's corona, visible in
images taken with a coronagraph or during a total solar eclipse. Coronal
streamers are thought to be associated with active regions and/or prominences
and are most impressive near the maximum of the solar cycle. They are large
scale magnetic structures of the equatorial solar corona.

Helmet/bipolar streamers are surrounded by open field of opposite polarities.

Coronal \textit{pseudostreamers} consist of multipolar closed field
surrounded by unipolar open magnetic field.

\subsection{Coronal loops}
\begin{itemize}
    \item Main observational feature of the magnetic structure in the
        upper solar atmosphere.
    \item \emph{Modelled} as flux tubes; probably consist of
        many flux tubes.
\end{itemize}

\subsection{Flares}
Rapid conversion of stored magnetic energy to particles acceleration and
excess radiation in the corona.

\subsection{Coronal Mass Ejections (CMEs)}
Release of magnetic energy; reach Earth in a few days.

\subsection{Jets}
Word for rapid burst of emission? Rapid upflows, plasma
ejections$\ldots$ UV spectral emission requires high temperatures.
For UV:\ flux from photospheric continuum is low. Chromosphere:
temp is lower and background is higher (compared to $\ldots$?)
so lines are in absorption. Low flux: radiative excitation doesn't occur.
High temps allow for collisional excitation and emission upon returning
to the ground state. $\Delta v = \sqrt{v_{th}^2+v_{Nth}^2}$.

\subsection{Solar cycle}
Three stages:
\begin{enumerate}
    \item Build-up of magnetic field
    \item Turnover (maximum)
    \item Reversal of poles
\end{enumerate}
This takes a total of $\sim$ 11 years, followed by another 11 years of the same
sequence, but with the poles reversed, for a total of 22 years.

\subsection{Solar wind}
Stream of energized, charged particles, primarily protons and electrons,
flowing outward from the sun at $v \leq 900$ km s$^{-1}$ and T = 10$^6$ K.
Solar wind plasma originates in thin, intense flux tubes at granule
and supergranule boundaries. Fast (steady) vs.\ slow (variable) wind.

\section{Magnetic field}
\begin{itemize}
    \item Strength and direction is hard to measure in the corona.
\end{itemize}

\section{Oscillations and waves}

\subsection{General}
\subsubsection{Instability}
A disturbance that is not stabilized by the resulting forces.
\subsubsection{Oscillations}
Three types:
\begin{enumerate}
    \item un-damped
    \item damped
    \item forced
\end{enumerate}
\subsubsection{Waves}
\begin{itemize}
    \item Caused by any turbulence in a medium.
    \item Observable properties
        \begin{itemize}
            \item Period
            \item wavelength
            \item amplitude
            \item temporal/spatial signatures (shape of perturbation)
            \item characteristic scenerios of evolution (e.g.\ damped?)
        \end{itemize}
\end{itemize}
\subsubsection{Standing waves}
Confined to a finite region of space $\rightarrow$ quantifies the frequency
\begin{itemize}
    \item Fundamental (2 nodes, one on each end):
        \[
            f_{1} = \frac{1}{2L}\sqrt{\frac{F}{\mu}}
            \]
        where $L$ is the string length, $F$ is the tension,
        $\mu$ is the mass density
    \item 3 nodes: $f_{2} = 2f_{1}$
    \item 4 nodes: $f_{3} = 3f_{1}$
\end{itemize}
\subsubsection{Propagating waves}
\subsubsection{wave number}
$k=2\pi=$ number of waves in a unit distance.

\subsubsection{Phase velocity}
\[
    v_{p} = \frac{\lambda}{T} = \frac{\omega}{k} = \frac{c}{n}
    \]
where $n$ = index of refraction ($n = 1$ in vacuum).
\[
    k = \frac{2\pi}{\lambda}, \omega = \frac{2\pi}{T}
    \]
\subsubsection{Group velocity}
Direction of a wave's energy flow (Poynting vector for light).
\[
    v_{g} = \frac{ \mathrm{d}\omega }{ \mathrm{d}k }
    \]
\subsubsection{Dispersion}
A mathematical term that applies to wave behavior; separation of a complex
wave into its component parts according to a given characteristic
(frequency, wavelength, etc.) ``Dissipation'' is a term that has to do with
shock and heating, and can be applied to anything. Safer to use this one
if you're not sure.
\textbf{Dispersion} is when the distinct phase velocities of the components of the
envelope cause the wave packet to ``spread out'' over time.
The components of the wave packet (or envelope) move apart to the degree
where they no longer combine to complete the envelope.
Causes different components of the wave to have different phase velocities.
Different kinds of dispersion:
\begin{itemize}
    \item Normal dispersion: strictly increasing Re ${\epsilon(\omega)}$ with
        increasing ${\omega}$.
    \item Anomalous dispersion: decreasing Re ${\epsilon(\omega)}$ with
        increasing ${\omega}$.
    \item Resonant absorption: occurs in regions where Im ${\epsilon(\omega)}$
        is large.
\end{itemize}
\begin{figure}[h]
    \centering
    \includegraphics[width=5in]{disp.png}
\end{figure}
No dispersion: ${v_{ph} = v_{gr}}$; dispersion: ${v_{ph} \neq v_{gr}}$.
A medium that is free from dispersion has index of refraction that is constant
as a function of frequency, so all wavelengths are similarly affected.
Permittivity and permeability are functions of frequency.
Non-dispersive waves have phase speed (${v_{ph}}$, speed that wave actually travels
through medium?) independent of wavenumber, ${k}$.
All waves of \emph{any} ${k}$ propagate at the same speed.
Uniform dissipation, resonant mode conversion, physical mechanisms vs. ?
Dispersion \emph{relation}:
\begin{itemize}
    \item Relates the wavelength (or wavenumber) of a wave to its
        frequency.
    \item Describe the effect of dispersion in a medium on the properties
        of a wave traveling through that medium.
\end{itemize}


\subsubsection{waveguide}
\begin{itemize}
    \item Width $\sim$ same order of magnitude as the wavelength of
        the guided wave.
    \item Pores extending up from the photosphere
        into the solar atmosphere can act as an
        MHD waveguide.
    \item Flux tubes are excellent waveguides
\end{itemize}


\subsection{Types of waves}
\subsubsection{acoustic waves (aka sound waves)}
\begin{itemize}
    \item Longitudinal
    \item Isotropic
    \item Compressible (propagate by means of adiabatic
        compression and decompression)
    \item travel at sound speed (determined by the medium in which
        they were \emph{created}, and maintain this speed even if they
        travel into medium with a different characteristic sound speed).
    \item Important quantities:
        \begin{itemize}
            \item sound pressure
            \item particle velocity
            \item particle displacement
            \item sound intensity
        \end{itemize}
    \item exist because of a pressure restoring force: local compression
        (or rarefaction) sets up a \emph{pressure gradient} in opposition
        to the motion
    \item carry energy away from source
    \item large enough amplitude $\rightarrow$ shock wave,
        but usually small; ambient gas slightly disturbed
    \item exist in medium with low or non-existent magnetic field
    \item isotropic \- propagate equally in all directions. The phase
        and group velocities are both equal to the sound speed (hence, the
        part where acoustic waves are sound waves).
\end{itemize}
\subsubsection{longitudinal waves}
waves in which the displacement of the \emph{medium} is in the
same direction as, or the opposite direction to,
the direction of travel of the wave.
\subsubsection{transverse waves}
\textcolor{magenta}{(from wikipedia:)} moving waves that \emph{consist} of
oscillations occurring perpendicular to the direction of energy transfer. If a
\emph{transverse} wave is moving in the positive x-direction, its
\emph{oscillations} are in up and down directions that lie in the y–z plane.
Light is an example of a transverse wave. ($\vec{B}$ and $\vec{E}$ oscillate in
directions perpendicular to the direction in which the light is actually
traveling). With regard to transverse waves in matter, the \emph{displacement
of the medium} is perpendicular to the direction of propagation of the wave.
Examples: A ripple in a pond and a wave on a string.

\subsection{Spherical harmonics}
3 kinds of resonant modes of oscillation:
\begin{itemize}
    \item p (pressure)
    \item g (gravity)
    \item f (?)
\end{itemize}
Numbers:
\begin{itemize}
    \item n - \emph{order}; Number of nodes in radial direction
    \item l - \emph{harmonic degree}; number of node lines on
        surface $\sim$ total number of planes slicing the sun.
    \item m - \emph{something}; number of surface nodal lines that
        cross the equator; phase\\
        $-l \leq m \leq l$ (direction of waves is important); number of
        planes slicing longitudinally.
\end{itemize}
\subsubsection{pressure waves}
P-modes are global acoustic oscillations.
Longitudinal, fast, generated by turbulence near the photosphere,
observed by measuring Doppler shift of absorption lines in the
photosphere, spherical harmonics:
\begin{itemize}
    \item $n$: radial order, number of nodes in radial direction
    \item $l$: angular degree, or harmonic degree,
        number of node lines on surface,
        $\sim$ total number of planes slicing the sun
    \item $m$: azimuthal number;
        number of surface nodal lines that cross the equator,
        number of planes slicing longitudinally. $-l\leq m\leq +l$
\end{itemize}
\subsubsection{gravity waves}
Generated in fluid medium or interface between two media when the
force of gravity or buoyency tries to restore equilibrium
e.g.\ ``wind waves'' from between atmosphere and ocean.

\subsection{Magnetohydrodynamic (MHD) waves}
\begin{itemize}
    \item study of electrically conducting fluids (plasma)
    \item Theoretical foundation:
        \begin{itemize}
            \item dispersion relation of MHD modes of a plasma cylinder
            \item models: loops, prominence fibrils, plumes, various filaments
            \item evolutionary equations
        \end{itemize}
    \item Considerations of observed waves:
        \begin{itemize}
            \item geometry: simple (slab or tube)? or more complex?
            \item mode
                \begin{itemize}
                    \item longitudinal vs.\ transverse
                    \item compressible vs.\ incompressible
                    \item oscillating vs.\ propagating
                    \item fast vs.\ slow
                    \item propagating vs.\ standing
                    \item isotropic vs.\ anisotropic
                    \item phase differences?
                \end{itemize}
        \end{itemize}
    \item \textcolor{red}{Interpreting observations: distinguish between modes
        that are pressure driven (acoustic/slow magnetoacoustic)
        and magnetically driven (Alfv\'en)}
    \item \textcolor{red}{Relations between the (internal and external)
        characteristic speeds (Alfv\'en, sound, and tube speeds)
        \emph{determine properties of MHD modes guided by the tube.}}
    \item The behavior of linear perturbations of the form
        $$ \delta P_{tot}(r)\textrm{exp}\left[i(k_zz+m\phi-\omega t)\right]  $$
        is governed by  the following system of first order differential
        equations and algebraic equations:
        $$ D\frac{d}{dr}(r\xi_r) = (C_A^2+C_s^2)\ldots  $$
    \item \textcolor{red}{Linearised equations}

\emph{Dissipation and Damping:}

    \item Dissipation of MHD waves is manifold:
        \begin{itemize}
            \item Couple with each other
            \item interact non-linearly
            \item resonantly interact with the closed waveguide
            \item devolop non-linearly (e.g.\ solitons or shock waves
                can form)
        \end{itemize}
    \item Inhomogeneous and magnetized plasma has two particular
        dissipation mechanisms of MHD waves:
        \begin{itemize}
            \item Resonant absorption
            \item Phase mixing
        \end{itemize}
\end{itemize}

\subsubsection{Characteristic speeds of MHD}
\begin{enumerate}
    \item Sound speed: $C_{s} = \sqrt{\frac{\gamma p_{0}}{\rho_{0}}}
            \approx 166 T_{o}^{1/2}$ m s$^{-1}$ = 200 km s$^{-1}$
    \item Alfv\'en speed: $C_{A} = \frac{B_{0}}{\sqrt{\mu_{0}\rho_{0}}}$
        \[
            V_{A} = 2.18\times10^{12}\frac{B_o}{\sqrt{n_o}}
            \textrm{m}\ \textrm{s}^{-1} = 3000 \textrm{km}\ \textrm{s}^{-1}
            \]
        (B$_{o} \sim$ 100 G; n$_{o}$ $\sim$ 10$^{16}$ m$^{-3}$)
    \item Tube/Cusp speed:
        $C_{T} = C_{s}C_{A}/\left(C_{A}^{2} + C_{s}^{2}\right)^{1/2}$
        = combination of sound and Alfv\'en speeds
\end{enumerate}

\subsubsection{Magnetoacoustic waves}
\begin{itemize}
    \item A magnetosonic wave (also magnetoacoustic wave) is a
        longitudinal wave of ions (and electrons) in a magnetized
        plasma propagating perpendicular to the stationary magnetic field.
    \item compressible
    \item slow MHD wave; slow MA waves only have 1-3 oscillations before
        damping out, observed oscillations are manifestation of rapid
        damping due to radiative energy losses. Reduced $\vec{B}$ regions
        = increased $\rho\rightarrow$ rapid radiative losses. \emph{Fast}
        MH waves radiate little because they're damped too slowly.
    \item collectively supported by the plasma environment, i.e., the wave mode acts
        across neighbouring magnetic field lines and across transverse
        plasma inhomogeneities.
    \item observed as disturbances of EUV (and possibly X-ray) emission
\end{itemize}
\subsubsection{kink modes}
Fast kink waves:
\begin{itemize}
    \item transverse (general property of fast waves)
    \item $v_{ph} = c_k = \sqrt{\frac{\rho_oV^2_{Ao}+\rho_eV^2_{Ae}}
        {\rho_o+\rho_e}} \approx V_A\sqrt{\frac{2}{1+\frac{\rho_e}{\rho_o}}} $
        in the low-$\beta$ plasma.
    \item Period $P=\frac{2\ell}{V_A}\sqrt{\frac{1+\rho_e/\rho_o}{2}}$
        where $\lambda=2\ell$ ($\ell$ is the loop length).
        Typically, $\ell \approx 60-600$ Mm in the corona.
    \item Period of global kink mode, $P = \frac{2\ell}{c_K}$
    \item Important observation from which magnetic field strength
        can be derived.
    \item slab: phase speed = V$_{A_e}$
    \item tube: kink speed:
        $$ c_K = \frac{B_i^2 + B_e^2}{\sqrt{\mu(\rho_i+\rho_e)}} $$
        (mean Alfv\'enic speed).
        Slab (or tube) is moved laterally; little variation in cross-sec,
        density, or intensity.
\end{itemize}
Standing and propagating both rapidly damped. Much more observations of
standing kink modes than propagating (`tadpole'-like structures moving
downward; open magnetic structures).

Slow kink waves:
\begin{itemize}
    \item longitudinal (velocity directed along magnetic field)
    \item compressible (variations in density and intensity)
\end{itemize}

Other (or simply unorganized):
\begin{itemize}
    \item oblique (inclined with respect to the flow direction)
    \item \emph{weakly compressible}, but could nevertheless be
        observed with imaging instruments as \emph{periodic standing}
        or \emph{propagating} displacements of coronal structures, e.g.\ coronal loops.
        The frequency of transverse or ``kink'' modes is given by the following expression:
            $$ w_K = \sqrt{ \frac{2k_zB^2}{\mu(\rho_i+\rho_e)}  }   $$
        In a cylindrical model of a loop,
        the parameter \emph{azimuthal wave number},
        m, is equal to 1 for kink modes.
    \item In the long wavelength limit, the phase speed of all but
        sausage fast modes tends to the so-called kink speed,
        which corresponds to the density weighed average Alfv\'en speed.
    \item Two instabilities of axisymmetric, current-carrying plasmas
        \begin{itemize}
            \item sawtooth relaxations
            \item fishbone oscillations
        \end{itemize}
        associated with instability of internal kink modes
    \item Both standing and propagating, T = $\sim$ seconds-minutes.
    \item lowest spatial harmonics along field; global (fundamental)
        modes of coronal loops nodes of displacement at footpoints,
        maximum at apex.
\end{itemize}
\subsubsection{sausage modes}
%Trapped fast modes supported by thick, dense loops because of the cutoff
%wavenumber (pfw_2). Observe spatially resolved radio (see sources in
%pfw_2).
\begin{itemize}
    \item m = 0
    \item The fast magnetoacoustic sausage mode is another type of
        localized, modified fast magnetoacoustic wave.
    \item Mainly transverse.
    \item Standing fast sausage modes have symmetric tube modes
    \item Has a long-wavelength cutoff (trapped sausage modes do
        not exist at longer wavelengths).
        Approaches a cut-off at the external Alfv\'en speed.
        (Under condition of \emph{mode localization}).
        \emph{Effects} of the sausage mode are easiest to observe in
        the radio regime (not the wave itself), where more of a
        ``point'' is observed, rather than extended$\ldots$
    \item Main feature is the periodic fluctuation of the cross-subsectional
        area of the waveguide. This change is also associated with
        periodic fluctuations in density and temperature within the
        waveguide.
    \item Distinct sign of sausage oscillations is when periodic
        phenomena in cross-subsection and intensity are almost
        180$^{\circ}$ out of phase $\rightarrow$ strong signal.
        (Less distinct signal is when periodicities in pore size
        don't match any intensity variations).
    \item Produce perturbations in density and magnetic field strength,
        and the corresponding plasma motions cause pulsations in the
        tube cross-subsection.
    \item Associated with perturbations of the loop cross-subsection
        and plasma concentration.
    \item Perturbations of plasma in the radial direction are stronger
        than perturbations along the field.
    \item Mode conversion and absorption through Alfv\'en resonance
        cannot take place (slow resonance can still operate).
    \item Phase speed is in the range between the Alfv\'en speed inside
        and outside the loop.
\end{itemize}
\subsubsection{slow magnetoacoustic waves}
\begin{itemize}
    \item $C_{T_{0}} < C_{slow} < C_{s_{0}}$
    \item longitudinal
    \item essentially acoustic in a low-$\beta$ plasma and subject to
        reflection if the frequency is below the cutoff frequency.
    \item propagate slower than both $V_{A}$ and $C_{s}$ (but faster than
        the cusp speed).
    \item Sound waves are subject to nonlinear steepening and shocking
        as the density decreases with height through the chromosphere.
\end{itemize}

\subsubsection{standing acoustic oscillations}
Characteristics:
\begin{itemize}
    \item Pressure forces in opposition
    \item Period = 7--31 minutes (20 minutes from another source)
    \item Decay times = 5.7--36.8 minutes
    \item Peak velocity = 200 km/sec
\end{itemize}
Standing oscillations vs.\ propagating waves
\begin{itemize}
    \item In loops, propagating waves damp before
        reaching opposite footpoint.
    \item Velocity and intensity are 90$^{\circ}$ out of phase
        for standing oscillations, and are in phase for propagating
        acoustic waves.
    \item Frequencies less than the cutoff are standing oscillations,
        waves with frequency greater than the cutoff propagate into
        the chromosphere.
    \item no loop shape change or displacement
    \item near footpoints.
\end{itemize}
\subsubsection{propagating acoustic waves (slow)}
\begin{itemize}
    \item $v_{ph}<150$ km s$^{-1}$ $\rightarrow$ slow
    \item longitudinal, compressive, anisotropic
    \item Parallel to $\vec{B}$, perturbation of $\vec{B}$ is negligible.
    \item Generated impulsively at one end of a footpoint.
    \item Only penetrate $\sim$ 10\% into loop before
        damped by thermal conduction
    \item weak dispersion in coronal conditions ($V_{A} \gg c_{s}$)
    \item 3 phases: periodic, QP, decay
    \item period = 3, 5, 10 minutes? Or 2--22 seconds? (see kink\_1),
    \item velocity: 50--200 km s$^{-1}$
    \item $c_{T} = \sqrt{\frac{c_{s}^2v_{A}^2}{c_{s}^2 + v_{A}^2}} $
        propagate sub-sonically at $c_{T}$, which is less than $c_{s}$
    \item ``large'' amplitude, max in top of chromosphere
    \item Observed using spectroscopy (intensity variations in
        EUV emission  and Doppler shifts)
\end{itemize}
\subsubsection{propagating acoustic waves (fast)}
\begin{itemize}
    \item $v_{ph}>150$ km s$^{-1}$ $\rightarrow$ fast
        (\emph{or} transverse standing waves).
    \item Quasi-isotropic
    \item Driven by magnetic forces + plasma pressure forces
    \item Compressive (magnetic sound wave)
    \item Speed: $c_{F} = \sqrt{c_{s}^2 + v_{A}^2} $
    \item Moreton waves in the chromosphere
    \item Fast EUV waves in the corona
\end{itemize}
\subsubsection{Alfv\'en waves}
Alfv\'en waves are a type of MHD wave in which ions oscillate in response to a
restoring force provided by an effective tension on the magnetic field lines.
Low frequency (compared to the ion cyclotron frequency) traveling
oscillation of the ion and the magnetic field. The ion mass
density provides the inertia and the magnetic field line
tension provides the restoring force.

They propagate in the direction of the magnetic field, although waves exist
at \emph{oblique} incidence and smoothly change into the \emph{magnetoacoustic}
wave when the propagation is \emph{perpendicular} to the magnetic field.
The motion of the ions and the perturbation of the magnetic
field are in the same direction and transverse to the direction of
propagation. The wave is dispersionless.

Locally supported, phase and group velocities with magnitude equal to
local Alfv\'en speed, directed along the magnetic field.
Alfv\'en waves are easily excited by various dynamical perturbations of
magnetic field lines, and are
weakly dissipative (can propagate long distances, deposit energy
and momentum far from source).

\begin{itemize}
    \item m$=$0 (Axisymmetric, or azimuthally symmetric)
    \item transverse (shear) perturbations (perpendicular to $\vec{F}_{res}$);
        plasma has characteristic elasticity.
    \item parallel to $\vec{B}$ (the \emph{group} velocity  is strictly along
        ${\vec{B}}$, but the \emph{phase} velocity doesn't have to be. The
        energy also propagates along the field lines, probably because the
        wave envelope is what carries all the information).
    \item (only) driving force: magnetic tension, the restoring force,
        which ``snaps'' the field back
        into a straight line, producing an Alfv\'en wave.
    \item Purely magnetic and incompressible in nature (in untwisted straight
        cylinder$\ldots$ twist may cause some compression).
    \item Displacement of plasma together with magnetic field frozen
        into it.
    \item velocity: $v_{A} = \frac{B}{\sqrt{\mu_{o}\rho}}$;
        $\sim$ 1000 km s$^{-1}$ in the corona.
    \item Mode conversion: fast MHD to Alfv\'en.
\end{itemize}
How to observe:
\begin{itemize}
    \item Only get Doppler shifts from \emph{long}-period waves
        ($>$ a few minutes).
    \item Measure additional (i.e.\ non-thermal) broadening
        of coronal emission lines; indirect way to observe short-period waves.
    \item Spatial variation in Doppler shift for long periods.
        Gyrosynchrotron emission in radio regime.
    \item $V_{A}$: temporal resolution?
\end{itemize}
Effects of twisting (or \emph{torsion}):
\begin{itemize}
    \item Coupling of various MHD modes
\end{itemize}

(From \emph{Priest}): Types of Alfv\'en waves:
\begin{itemize}
    \item Shear or Torsional: No accompanying pressure or
        density changes (plasma).
    \item Compressional or Fast-Mode: Becomes a fast
        magnetoacoustic or fast-mode wave when pressure gradients
        are included (Note: these are often the preferred names,
        even when the pressure gradient is unimportant, so as to
        avoid confusion with shear Alfv\'en waves).
\end{itemize}
\subsubsection{EIT wave}
\subsubsection{Moreton wave}
Chromospheric signature of a large-scale coronal shock wave. Generated
by flares, $\sim$ fast-mode MHD waves.

\subsection{resonance}
Periodic driving force frequency matches the wave frequency
$\rightarrow$ large amplitude.
\subsection{resonant absorption}
\begin{itemize}
    \item Mechanism of wave heating
    \item could damp kink mode oscillations.
    \item Loss of acoustic power in sunspots. Time scales:
        \begin{enumerate}
            \item damping: collective mode $\rightarrow$ local mode,
                \emph{independent} of dissipation.
            \item dissipative damping of small scale perturbations of local
                mode.
        \end{enumerate}
        $\tau_1 \ll \tau_2$
    \item inherently non-linear
\end{itemize}





\subsection{torsional vibration}
\begin{itemize}
    \item angular vibration of an object $\sim$ shaft along the
        axis of vibration
\end{itemize}

\subsection{Modes}
A wave may be a superposition of lots of other waves. Each of those
waves is a ``mode'' of the resultant wave (think of the foundation
of Fourier Analysis: sums of sines and cosines).
Modes with the lowest wave number are \emph{global}, or
\emph{fundamental} modes.
\begin{itemize}
    \item Different modes are driven by \emph{different restoring forces}.
\end{itemize}
\subsection{Normal modes}
Vibrational state of an oscillatory \emph{system} where the frequency
is the same for all elements. E.g.\ resonant frequencies: equally
spaced multiples of the fundamental.
\subsection{trapped modes}
For a specified geometry, uniqueness of the solution to a forcing problem
at a particular frequency is equivalent to the non-existence of a trapped
mode at that frequency. A trapped mode is a solution of the corresponding
homogeneous problem and represents a free oscillation with finite energy
of the fluid surrounding the fixed structure. For a given structure,
trapped modes may exist only at discrete frequencies.
Mathematically, a trapped mode corresponds to an eigenvalue embedded
in the continuous spectrum of the relevant operator.
\subsection{flute modes}
See \emph{ballooning modes}
\subsection{ballooning modes}
\begin{itemize}
    \item $m > 1$
    \item Role not established yet
\end{itemize}
\subsection{leaky modes}
\begin{itemize}
    \item Waves are allowed to radiate into the external medium,
        i.e.\ the condition of mode localization is relaxed.
    \item complex eigenfrequencies eigenfrequency
    \item Bessel functions are replaced by Hankel functions in the
        dispersion relation. Can be the fundamental harmonic.
    \item Wavenumbers below cutoff value?
    \item Has electric field that decays monotonically for a finite
        distance in the transverse direction but becomes oscillatory
        everywhere beyond that finite distance.
    \item Mode ``leaks'' out of
        the waveguide as it travels down it, producing attenuation.
    \item Relative amplitude of oscillatory part (leakage rate)
        must be sufficiently small that the mode maintains its
        shape as it decays, in order to be called a mode at all.
\end{itemize}
\subsection{Global/Fundamental modes}
\begin{itemize}
    \item $y(x,t)$ governed by some PDE with no \emph{explicit} time dependence.
    \item A global mode is a solution of the form
        $y(x,t) = \hat{y}(x)e^{i\omega t}$
    \item PDE-dynamical system of infinitely many equations coupled together.
\end{itemize}
\subsection{Surface modes}
evanescent behavior in both media (inside and outside cylinder).
In the context of flux tubes and coronal seismology, surface waves are
non-oscillatory inside the flux tube (body waves are oscillatory).
\subsection{Body modes}
In the context of flux tubes and coronal seismology,
body waves are oscillatory inside the tube (whereas surface waves are
non-oscillatory inside the tube).
\subsection{mode conversion}
\subsection{mode coupling}
\subsection{phase mixing}
\begin{itemize}
    \item Cause decay of Alfv\'en waves, which suffer intense phase mixing
    \item Large gradients in Alfv\'{e}n velocity.
    \item Part of resonant absorption, but is effective over entire range of frequencies,
        not just resonant ones; spread over volume rather than localized in
        narrow resonance layer.
    \item Occurs spatially (propagating wave) or in time (standing wave).
        \begin{itemize}
            \item Space: produces effective wavenumber $k_{x}^{*}$ which increases
                with $z$, so that the effective wavelength decreases. Eventually
                dissipation converts energy to heat.
                \[
                    k_{x}^{*} = \frac{\mathrm{d}k_{z}}{\mathrm{d}x}z
                    \]
            \item Time: field lines become more and more out of phase.
        \end{itemize}
        \emph{Not} likely to operate in closed magnetic structures
        (e.g.\ coronal loops)
\end{itemize}


\section{Observables}

\subsection{Polarimetry}

Circular vs. linear polarization\ldots
linear is $\sim$100 times brighter than circular.
\begin{itemize}
    \item Doesn't apply to longitudinal waves, e.g.\ sound.
    \item Linear waves oscillate (transversely) in a single direction.
\end{itemize}
\subsection{Zeeman effect}
\subsection{Hanle effect}
\begin{itemize}
    \item Reduction in polarization of light (depolarization)
    \item Used to indirectly measure $\overline{B}$
    \item Depolarization of Fe XIII 10747 in corona (originally polarized due
        to scattering of unpolarized light from photosphere) preserves
        information about the direction, but not the strength, of the magnetic
        field.
\end{itemize}
\subsection{Emission measure}
\begin{itemize}
    \item $C_{A_{0}} < C_{fast} < C_{A_{e}} $
    \item highly dispersive.
    \item Kink modes
    \item Sausage modes
    \item propagate faster than both $V_{A}$ and $C_{s}$
\end{itemize}
\subsection{Differential Emission Measure (DEM)}
\[
    EM = \int{n_{e}^{2}\mathrm{d}V}
    \]
\[
    DEM = \xi(T_{e}) = \int{\frac{n^{2}(r)}{|\nabla{T}|}\mathrm{d}S_{T}}
    \]
or
\[
    DEM = \frac{\mathrm{d}}{\mathrm{d}T}\left[\int{n_{e}^{2}\mathrm{d}S}\right]
    \]
in units of [cm ${^{-5}}$ K${^{-1}}$]. DEM is the integral of electron
density squared along an optically thin line of sight as a function of
temperature.
``Can predict the coronal bremsstrahlung flux component\ldots
bremsstrahlung correlates with EUV emission.''

\subsection{Fraunhofer lines}
First observed in 1802 by William Hyde Wallaston, and carefully
studied by Joseph von Fraunhofer (1787--1826), who catalogued
the wavelengths in 1814. Ionized sodium: inert gas structure,
no lines in visible regime. Dark absorption features in solar
spectrum, in cooler layers of the atmosphere.
\begin{itemize}
    \item H and K lines of ionized Calcium
    \item D line of neutral sodium (5890 \AA{}, 5896 \AA{};
        resonance lines between n=1 and n-2).
    \item E (iron)
    \item C and F (hydrogen)
\end{itemize}

\subsection{gyrosynchrotron radiation}
Electromagnetic emission emitted by mildly relativistic electrons moving
in a magnetic field
(as opposed to synchrotron, with \emph{ultra}relativistic particles).


\section{Analysis techniques}
\subsection{periodogram}
\subsection{wavelet analysis}
\subsection{WKB approximation}


\section{Maths}
\subsection{Bessel's equations}

(From Boas p. 587): Bessel functions are damped sines and cosines.
Solutions of differential equations can be represented by power series.
Graphs, formulas; electricity, heat, hydrodynamics, elasticity, wave motion,
quantum$\ldots$, cylindrical symmetry$\ldots$

Bessel's differential equation:
\[
    x^{2}\frac{\mathrm{d}^{2}y}{\mathrm{d}x^{2}} +
    x\frac{\mathrm{d}y}{\mathrm{d}x} +
    \left( x^{2}-p^{2} \right) y
    = 0
    \]
where ${p}$ is the \emph{order} of the Bessel function ${y}$ and is a constant, and
${y}$ is the solution.
${a}$ = integer $\rightarrow$ cylindrical,
${a}$ = half-integer $\rightarrow$ spherical.
\[
    x^{2}y'' + xy' + \left( x^{2}-p^{2} \right) y = 0
    \]
\[
    x\left(xy'\right) + \left(x^{2}-p^{2} \right) y' = 0
    \]

\subsection{Fourier analysis}

\section{Other}
\subsection{adiabatic index}
The adiabatic index is the ratio of specific heats:
\[
    \gamma = \frac{c_{P}}{c_{V}}
    \]

\section{Solar instruments}
\subsection{SoHO}
\emph{The Solar and Heliospheric Observatory}

Instruments:
\begin{itemize}
    \item \emph{EUV Imaging Telescope} (EIT) discovered EUV waves
        in the corona.
\end{itemize}

\subsection{DKIST}
\begin{description}
    \item [Visible Broadband Imager (VBI)] high spatial/temporal resoloution
        of the atmosphere.
    \item [Visible Spectro-Polarimeter (ViSP)] multi-line spectropolarimeter
    \item [double Fabry-P\'erot based Visible Tunable Filter (VTF)] high spatial
        resolution spectropolarimetry
    \item [DL-NIRSP] fiber-fed diffraction-limited NIR spectropolarimeter
    \item [Cryo-NIRSP] cryogenic NIR SP for coronal magnetic field measurements
        and on-disk observations of, e.g., the CO line at 4.7 ${\mu}$
\end{description}
\subsection{Hinode}

\subsection{Interface Region Imaging Spectrograph (IRIS)}
\begin{itemize}
    \item Lockheed Martin Solar and Astrophysical Laboratory (LMSAL)
    \item Launched June 2013; sun-synchronous, low-Earth orbit.
    \item Goals: Understand how chromosphere is energized
    \item Revealed complexity, density/temperature contrasts in the interface
        region.
    \item 19 cm Cassegrain telescope
        \begin{itemize}
            \item dual-range UV spectrograph (imaging) with 1 second cadence,
                0.3'' spatial resolution, $<$ 1\AA{} spectral resolution.
            \item slit-jaw imager (SJI) with four passbands:
                \begin{enumerate}
                    \item CII 1335\AA{} (transition region line)
                    \item SiIV 1403\AA{} (transition region line)
                    \item Mg II k 2796\AA{} (chromospheric line)
                    \item 2830\AA{} (photospheric passband)
                \end{enumerate}
        \end{itemize}
    \item ``NASA Small Explorer developed and operated by LMSAL with mission operations
    executed at NASA Ames Research center and major contributions to downlink
    communications funded by ESA and the Norwegian Space Center.''
    (Bryans et al. 2016)
\end{itemize}
\subsection{RHESSI}

\subsection{Skylab}
Launched 1973

Instruments:
\begin{itemize}
    \item Apollo Telescope Mount (ATM)
\end{itemize}

\subsection{Coronal Multichannel Polarimeter (CoMP)}
\begin{itemize}
    \item Daily observations of coronal emission-line polarization.
    \item 20 cm
\end{itemize}

\subsection{STEREO}
\subsection{Solar Dynamics Observatory (SDO)}
\subsubsection{AIA}
\begin{table}[h]
    \centering
    \begin{tabular}{c c c c}
        Wavelength (\AA{}) & Ion(s) & Height & Temperature (K)\\
        \hline
        193 & Fe & 1 R$_{\odot}$ & $10^{6}$\\
        \hline\hline
    \end{tabular}
    \caption{Properties of AIA bassbands}
\end{table}
\subsubsection{EVE}
\subsubsection{HMI}
\subsection{TRACE}
The Solar Terrestrial Relations Observatory (STEREO) is a solar observation
mission. Two nearly identical spacecraft were launched in 2006 into
orbits around the Sun that cause them to respectively pull farther ahead
of (STEREO A)
and fall gradually behind (STEREO B) the Earth. This enables stereoscopic
imaging of the Sun and solar phenomena, such as coronal mass ejections.

Instruments:
\begin{itemize}
    \item Sun Earth Connection Coronal and Heliospheric Investigation
        (SECCHI) has five cameras: an extreme ultraviolet imager (EUVI),
        two white-light coronagraphs (COR1 and COR2), and two heliospheric
        imagers (called HI1 and HI2). The first three telescopes are
        collectively known as the Sun Centered Instrument Package (SCIP),
        and image the solar disk and the inner and outer corona. HI1 and
        HI2 image the space between Sun and Earth. The purpose of SECCHI is
        to study the 3-D evolution of Coronal Mass Ejections through their
        full journey from the Sun's surface through the corona and
        interplanetary medium to their impact at Earth.
    \item In-situ Measurements of Particles and CME Transients (IMPACT) will
        study energetic particles, the three-dimensional distribution of
        solar wind electrons and interplanetary magnetic field.
    \item PLAsma and SupraThermal Ion Composition (PLASTIC) will study the
        plasma characteristics of protons, alpha particles and heavy
        ions.
    \item STEREO/WAVES (SWAVES) is a radio burst tracker that will study
        radio disturbances traveling from the Sun to the orbit of
        Earth.
\end{itemize}




\end{document}
