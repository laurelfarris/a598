\documentclass[12pt]{article}
\usepackage[margin=1in]{geometry}
\usepackage{titlesec}
\usepackage{mdwlist}
\setlength{\parindent}{0em}
\setlength{\parskip}{0.5em}
\usepackage{enumitem}
%\setlist[1]{itemsep=-2pt}

\titleformat*{\section}{\Large\bfseries}
\titleformat*{\subsection}{\large\bfseries}

\title{\vspace{-0.5in}Sun Stuff}
\author{}
\date{}

\begin{document}
\maketitle

\vspace{-1in}

\section*{A}

\subsection*{active regions}

\subsection*{Alfv\'en waves}

\section*{C}

\subsection*{coronal holes}

\subsection*{coronal loops}
\begin{itemize*}
    \item \emph{Modelled} as flux tubes; probably consist of
        many flux tubes.
\end{itemize*}

\subsection*{coronal mass ejections (CMEs)}

\subsection*{faculae}
    \begin{itemize*}
        \item Appear in \emph{photosphere}; same thing as plages.
        \item Bright spots - reason why total brightness is higher at
        solar maximum.
    \end{itemize*}

\subsection*{filament}
    \begin{itemize*}
        \item Viewed on disk; same thing as prominences.
        \item Thin, cool, dark ribbons
    \end{itemize*}

\subsection*{flares}

\subsection*{flux tubes}
    \begin{itemize*}
        \item Formed deep in the convection zone.
        \item Rise by magnetic buoyancy in an $\Omega$-shaped loop.
        \item Magnetic field lines can be thought of as infinitely
          thin flux tubes.
    \end{itemize*}

\subsection*{frozen-in flux}
In a perfectly conducting material (i.e.\ $\eta = 0$),
Ohm's law goes from
$ \vec{E} + \vec{v} \times \vec{B} = \vec{J}\eta $ to
$ \vec{E} + \vec{v} \times \vec{B} = 0 $
Nothing can be perpendicular to the field lines $\ldots$
See Alfv\'en's Theorem.

\section{J}

\subsection*{jets}

\section{K}
\section{L}
\section{M}
\section{N}
\section{O}

\section{P}

\subsection*{plages}
    \begin{itemize*}
        \item Appear in \emph{chromosphere}; same thing as faculae.
    \end{itemize*}

\subsection*{pores}

\subsection*{prominence}
    \begin{itemize*}
        \item Viewed on the limb; same thing as filaments.
        \item May erupt sometime during its life and be associated
        with a CME
    \end{itemize*}

\section{Q}
\section{R}

\section{S}

\subsection*{solar wind}

\subsection*{spicules}

\subsection*{sunspots}
Dark regions of intense magnetic field.

\subsection*{surges}

\end{document}
