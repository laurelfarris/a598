\documentclass[12pt]{article}
\usepackage[margin=1in]{geometry}
\usepackage{titlesec}
\usepackage{mdwlist}
\setlength{\parindent}{0em}
\setlength{\parskip}{0.5em}
\usepackage{enumitem}
%\setlist[1]{itemsep=-2pt}

\titleformat*{\section}{\large\bfseries}

\title{\vspace{-0.5in}Sun Stuff}
\author{}
\date{}

\begin{document}
\maketitle

\vspace{-1in}

\paragraph{active regions}
\paragraph{Alfv\'en waves}
\paragraph{coronal holes}
\paragraph{coronal loops}
\paragraph{coronal mass ejections (CMEs)}
\paragraph{faculae}
\paragraph{filament}
\paragraph{flares}
\paragraph{flux tubes} Magnetic field lines can be thought of as infinitely
thin flux tubes.
\paragraph{frozen-in flux}
In a perfectly conducting material (i.e.\ $\eta = 0$),
Ohm's law goes from
$ \vec{E} + \vec{v} \times \vec{B} = \vec{J}\eta $ to
$ \vec{E} + \vec{v} \times \vec{B} = 0 $
Nothing can be perpendicular to the field lines $\ldots$
See Alfv\'en's Theorem.

\paragraph{plages}
\paragraph{pores}
\paragraph{prominence}
\paragraph{solar wind}
\paragraph{spicules}
\paragraph{sunspots}

\end{document}
