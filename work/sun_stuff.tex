\documentclass[12pt]{article}
\usepackage[margin=1in]{geometry}
\usepackage{titlesec}
\usepackage{mdwlist}
\setlength{\parindent}{0em}
\setlength{\parskip}{0.5em}
\usepackage{enumitem}
%\setlist[1]{itemsep=-2pt}

\titleformat*{\section}{\Large\bfseries}
\titleformat*{\subsection}{\large\bfseries}

\title{\vspace{-0.5in}Sun Stuff}
\author{}
\date{}

\begin{document}
\maketitle

\vspace{-1in}

\section*{A}

\subsection*{active regions}

\subsection*{Alfv\'en waves}

\section*{C}

\subsection*{coronal holes}

\subsection*{coronal loops}
\begin{itemize*}
    \item \emph{Modelled} as flux tubes; probably consist of
        many flux tubes.
\end{itemize*}

\subsection*{coronal mass ejections (CMEs)}

\subsection*{faculae}
    \begin{itemize*}
        \item Appear in \emph{photosphere}; same thing as plages.
        \item Bright spots --- reason why total brightness is higher at
            solar maximum.
        \item small scale bright points in the vicinity of sunspots
    \end{itemize*}

\subsection*{filament}
    \begin{itemize*}
        \item Viewed on disk; same thing as prominences.
        \item Thin, cool, dark ribbons
    \end{itemize*}

\subsection*{flares}

\subsection*{flux tubes}
    \begin{itemize*}
        \item Formed deep in the convection zone.
        \item Rise by magnetic buoyancy in an $\Omega$-shaped loop.
        \item Magnetic field lines can be thought of as infinitely
          thin flux tubes.
    \end{itemize*}

\subsection*{frozen-in flux}
In a perfectly conducting material (i.e.\ $\eta = 0$),
Ohm's law goes from
$ \vec{E} + \vec{v} \times \vec{B} = \vec{J}\eta $ to
$ \vec{E} + \vec{v} \times \vec{B} = 0 $
Nothing can be perpendicular to the field lines $\ldots$
See Alfv\'en's Theorem.

\section{J}
\subsection*{jets}
Word for rapid burst of emission? Rapid upflows, plasma
ejections$\ldots$

\section{K}
\section{L}
\section{M}
\section{N}
\section{O}

\section{P}
\subsection*{plage}
    \begin{itemize*}
        \item Appear in \emph{chromosphere}; same thing as faculae.
    \end{itemize*}

\subsection*{plume}
Apparently help to shield Earth from solar storms.
They are long thin streamers that project outward from the Sun's north and
south poles. We often find bright areas at the footpoints of these
features that are associated with small magnetic regions on the solar
surface. These structures are associated with the \emph{open} magnetic
field lines at the Sun's \emph{poles}. The plumes are formed by the action of
the solar wind in much the same way as the peaks on the helmet
streamers.

\subsection*{pores}
Small sunspots with an umbra, but no penumbra.
Size on order of upper limit of magnetic bright points ($\sim$ 1700
km).

\subsection*{prominence}
    \begin{itemize*}
        \item Viewed on the limb; same thing as filaments.
        \item May erupt sometime during its life and be associated
        with a CME
    \end{itemize*}

\section{Q}
\section{R}
\section{S}
\subsection*{solar wind}
\subsection*{spicules}
Dynamic jet of about 500 km diameter in the \emph{chromosphere} of the Sun.
It moves upward at about 20 km/s from the photosphere.

\subsection*{sunspots}
Dark regions of intense magnetic field.

\subsection*{surges}

\end{document}
