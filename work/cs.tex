\documentclass[12pt]{article}
\usepackage[margin=1in]{geometry}
\usepackage{titlesec}
\setlength{\parindent}{0em}
\setlength{\parskip}{0.5em}
\usepackage{enumitem}
\usepackage{mdwlist}
\usepackage{color}

\title{\vspace{-0.5in}Wave Stuff}
\author{}
\date{}

\begin{document}
\maketitle

\vspace{-1in}

\section*{A}

\subsection*{acoustic waves (aka sound waves)}
\begin{itemize*}
    \item Longitudinal
    \item propagate by means of adiabatic compression and decompression
    \item travel at sound speed (determined by medium)
    \item Important quantities:
        \begin{itemize*}
            \item sound pressure
            \item particle velocity
            \item particle displacement
            \item sound intensity
        \end{itemize*}
    \item exist because of a pressure restoring force: local compression
        (or rarefaction) sets up a \emph{pressure gradient} in opposition
        to the motion
    \item carry energy away from source
    \item large enough amplitude $\rightarrow$ shock wave,
        but usually small; ambient gas slightly disturbed
    \item exist in medium with low or non-existent magnetic field
    \item isotropic
\end{itemize*}

\subsection*{Alfv\'en waves}
\begin{itemize*}
    \item transverse (perpendicular to $\vec{F}_{res}$)
    \item torsional
    \item incompressible
    \item type of MHD wave in which ions oscillate in response to a restoring
        force provided by an effective tension on the magentic field lines.
    \item low frequency (compared to the ion cyclotron frequency) traveling
        oscillation of the ion and the magnetic field.
    \item propagates in the direction of the magnetic field, although waves exist
        at oblique incidence and smoothly change into the magnetosonic wave
        when the propagation is perpendicular to the magnetic field.
    \item possibly contribute to ion heating and acceleration in the solar
        wind via ion-cyclotron resonance with $ \nu > 10$ Hz, where R$_m$ is
        very large; small scale structures down to resistive or kinetic scales.
    \item may be generated by tearing or interchange instabilities.
    \item locally supported, phase and group velocities with magnitude equal to
        local Alfv\'en spped, directed along the magnetic field.
    \item weakily dissipative (can propagate long distances, deposit energy
        and momentum far from source).
    \item easily excited by various dynamical perturbations of magnetic field
        lines.
    \item promising tool for heating and diagnositcs of coronal magnetic structures.
    \item magnetic force \emph{is} the restoring force$\ldots$ ``snaps'' back
        into a straight line, producing an Alfv\'en wave
    \item velicity is greater than the local sound speed in coronal conditions.
    \item questions/things to look up:
        \begin{itemize*}
            \item how often are they produced? how many?
            \item present in magetnic structures? or everywhere?
            \item three types (shear, torsional, fast magnetoacoustic)?
        \end{itemize*}
\end{itemize*}

\section*{B}

\subsection*{ballooning modes}
\begin{itemize*}
    \item m $>$ 1
    \item Role not established yet
\end{itemize*}

\subsection*{body modes}

\section*{C}

\subsection*{coronal loops}
\begin{itemize*}
    \item Main observational feature of the magnetic structure in the
        upper solar atmosphere.
\end{itemize*}

\section*{D}

\subsection*{dispersion relations}
\begin{itemize*}
    \item Relates the wavelength (or wavenumber) of a wave to its
        frequency.
    \item Describe the effect of dispersion in a medium on the properties
        of a wave traveling through that medium.
\end{itemize*}

\section*{E}

\section*{F}

\subsection*{fast waves}
\begin{itemize*}
    \item $C_{A_0} < C_{fast} < C_{A_e} $
    \item highly dispersive.
    \item Kink modes
    \item Sausage modes
    \item propagate faster than both $V_A$ and $C_s$
\end{itemize*}

\section*{G}

\subsection*{gravity waves}
Generated in fluid medium or interface between two media when the
force of gravity or buoyency tries to restore equilibrium
e.g.\ ``wind waves'' from between atmosphere and ocean.
P-modes are global acoustic oscillations.
(Note: \emph{gravitational} waves are not the same thing; they
have something to do with relativity).

\subsection*{group velocity}

\subsection*{gyrosynchrotron radiation}
Electromagnetic emission emitted by mildly relativistic electrons moving
in a magnetic field
(as opposed to synchrotron, with \emph{ultra}relativistic particles).

\section*{H}

\section*{I}
\subsection*{instabilities}
A disturbance that is not stabilized by the resulting forces.

\section*{J}
\section*{K}

\subsection*{kink modes/waves/oscillations}
\textcolor{magenta}{(from wikipedia:)}
Kink (or transverse) modes,
are oblique (inclined with respect to the flow direction)
fast magnetoacoustic
(also known as magnetosonic waves) guided by the plasma structure;
the mode \emph{causes the displacement of the axis of the plasma structure}.
These modes are \emph{weakly compressible}, but could nevertheless be
observed with imaging instruments as \emph{periodic standing}
or \emph{propagating} displacements of coronal structures, e.g.\ coronal loops.
The frequency of transverse or ``kink'' modes is given by the following expression:
    $$ w_K = \sqrt{ \frac{2k_zB^2}{\mu(\rho_i+\rho_e)}  }   $$
In a cylindrical model of a loop,
the parameter \emph{azimuthal wave number},
m, is equal to 1 for kink modes.
This means that the cylinder is ``swaying with fixed ends''$\ldots$?

In the long wavelength limit, the phase speed of all but
sausage fast modes tends to the so-called kink speed,
which corresponds to the density weighed average Alfv\'en speed.

\section*{L}

\subsection*{leaky modes}
\begin{itemize*}
    \item Waves are allowed to radiate into the external medium,
        i.e.\ the condition of mode localization is relaxed.
    \item Bessel functions are replaced by Hankel functions in the
        dispersion relation. Can be the fundamental harmonic.
    \item Wavenumbers below cutoff value?
    \item Has electric field that decays monotonically for a finite
        distance in the transverse direction but becomes oscillatory
        everywhere beyond that finite distance.
    \item Mode ``leaks'' out of
        the waveguide as it travels down it, producing attenuation.
    \item Relative amplitude of oscillatory part (leakage rate)
        must be sufficiently small that the mode maintains its
        shape as it decays, in order to be called a mode at all.
\end{itemize*}

\subsection*{longitudinal waves}
waves in which the displacement of the \emph{medium} is in the
same direction as, or the opposite direction to,
the direction of travel of the wave.

\section*{M}
\subsection*{magnetoacoustic waves}
A magnetosonic wave (also magnetoacoustic wave) is a longitudinal wave
of ions (and electrons) in a magnetized plasma propagating perpendicular
to the stationary magnetic field.
Magnetoacoustic modes are collectively supported by the plasma
environment, i.e., the wave mode acts across neighbouring magnetic field
lines and across transverse plasma inhomogeneities.
\begin{itemize*}
    \item compressible
    \item slow?
\end{itemize*}

\subsection*{magnetohydrodynamic (MHD) waves}
\begin{itemize*}
    \item study of electrically conducting fluids (plasma)
    \item Dissipation of MHD waves is manifold:
        \begin{itemize*}
            \item Couple with each other
            \item interact non-linearly
            \item resonantly interact with the closed waveguide
            \item devolop non-linearly (e.g.\ solitons or shock waves
                can form)
        \end{itemize*}
    \item Inhomogeneous and magnetized plasma has two particular
        \emph{dissipation} mechanisms of MHD waves:
        \begin{itemize*}
            \item Resonant absorption
            \item Phase mixing
        \end{itemize*}
    \item Theoretical modeling:
        \begin{itemize*}
            \item dispersion relations
            \item evolutionary equations
        \end{itemize*}
    \item Theoretical foundation:
        \begin{itemize*}
            \item dispersion reslatiosn of MHD modes of a plasma cylinder
            \item models: loops, prominence fibrils, plumes, various filaments
            \item evolutionary equations
        \end{itemize*}
    \item Considerations of observed waves:
        \begin{itemize*}
            \item geometry: simple (slab or tube)? or more complex?
            \item mode
                \begin{itemize*}
                    \item longitudinal vs.\ transverse
                    \item compressible vs.\ incompressible
                    \item oscillating vs.\ propagating
                    \item fast vs.\ slow
                    \item propagating vs.\ standing
                    \item isotropic vs.\ anisotropic
                    \item phase differences?
                \end{itemize*}
            \item damping: leakage by radiating modes? resonant abs? phase mixing?
            \item corona: compare obs and theory to discover new things
        \end{itemize*}
\end{itemize*}


\subsection*{modes}
A wave may be a superposition of lots of other waves. Each of those
waves is a ``mode'' of the resultant wave (think of the foundation
of Fourier Analysis: sums of sines and cosines).
Modes with the lowest wave number are \emph{global}, or
\emph{fundamental} modes.
\begin{itemize*}
    \item Different modes are driven by \emph{different restoring forces}.
\end{itemize*}

\subsection*{Moreton wave}
Chromospheric signature of a large-scale coronal shock wave. Generated
by flares, $\sim$ fast-mode MHD waves.

\section*{N}
\subsection*{Normal modes}
Vibrational state of an oscillatory \emph{system} where the frequency
is the same for all elements. E.g.\ resonant frequencies: equally
spaced multiples of the fundamental.

\section*{O}
\subsection*{oscillations}
Three types:
\begin{enumerate*}
    \item un-damped
    \item damped
    \item forced
\end{enumerate*}

\section*{P}

\subsection*{phase mixing}
\begin{itemize*}
    \item Large gradients in Alfv\'en velocity.
    \item Alfv\'en waves suffer intense phase mixing
    \item Cause decay of Alfv\'en waves.
    \item \emph{Not} likely to operate in closed magnetic structures
        (e.g.\ coronal loops)
\end{itemize*}

\subsection*{Phase speed}
$$ v_p = \frac{\lambda}{T} = \frac{\omega}{k} $$
$$ k = \frac{2\pi}{\lambda}, \omega = \frac{2\pi}{T} $$

\subsection*{polarization}
\begin{itemize*}
    \item Doesn't apply to longitudinal waves, e.g.\ sound.
    \item Linear waves oscillate (transversely) in a single direction.
\end{itemize*}

\subsection*{pressure waves}
This is what pressure waves are $\ldots$

\section*{R}
\subsection*{resonant absorption}
Mechanism of wave heating that could damp kink mode oscillations.
Loss of acoustic power in sunspots. Time scales:
\begin{enumerate*}
    \item damping: collective mode $\rightarrow$ local mode,
        \emph{independent} of dissipation.
    \item dissipative damping of small scale perturbations of local
        mode.
\end{enumerate*}
$\tau_1 \ll \tau_2$

\section*{S}
\subsection*{sausage modes}
\begin{itemize*}
    \item m = 0
    \item The fast magnetoacoustic sausage mode is another type of
        localized, modified fast magnetoacoustic wave.
    \item Mainly transverse.
    \item Standing fast sausage modes have symmetric tube modes
    \item Has a long-wavelength cutoff (trapped sausage modes do
        not exist at longer wavelengths).
        Approaches a cut-off at the external Alfv\'en speed.
    \item Main feature is the periodic fluctuation of the cross-sectional
        area of the waveguide. This change is also associated with
        periodic fluctuations in density and temperature within the
        waveguide.
    \item Distinct sign of sausage oscillations is when periodic
        phenomena in cross-section and intensity are almost
        180$^{\circ}$ out of phase $\rightarrow$ strong signal.
        (Less distinct signal is when periodicities in pore size
        don't match any intensity variations).
    \item Produce perturbations in density and magnetic field strength,
        and the corresponding plasma motions cause pulsations in the
        tube cross-section.
    \item Associated with perturbations of the loop cross-section
        and plasma concentration.
    \item Perturbations of plasma in the radial direction are stronger
        than perturbations along the field.
    \item Mode conversion and absorption through Alfv\'en resonance
        cannot take place (slow resonance can still operate).
    \item Phase speed is in the range between the Alfv\'en speed inside
        and outside the loop.
\end{itemize*}

\subsection*{slow waves}
\begin{itemize*}
    \item $C_{T_0} < C_{slow} < C_{s_0}$
    \item longitudinal
    \item acoustic(?)
    \item propagate slower than both $V_A$ and $C_s$
\end{itemize*}

\subsection*{speeds}
Characteristic speeds of MHD are the sound speed and Alfv\'enic speed,
given by:
$ C_s = \sqrt{\frac{\gamma p_0}{\rho_0}} $ and
$ C_A = \frac{B_0}{\sqrt{\mu_0\rho_0}} $

\subsection*{spherical harmonics}
3 kinds of resonant modes of oscillation:
\begin{itemize*}
    \item p (pressure)
    \item g (gravity)
    \item f (?)
\end{itemize*}
Numbers:
\begin{itemize*}
    \item n - \emph{order}; Number of nodes in radial direction
    \item l - \emph{harmonic degree}; number of node lines on
        surface $\sim$ total number of planes slicing the sun.
    \item m - \emph{something}; number of surface nodal lines that
        cross the equator; phase\\
        $-l \leq m \leq l$ (direction of waves is important); number of
        planes slicing longitudinally.
\end{itemize*}

\subsection*{surface modes}
evanescent behavior in both media (inside and outside cylinder).


\section*{T}

\subsection*{torsional vibration}
\begin{itemize*}
    \item angular vibration of an object $\sim$ shaft along the
        axis of vibration
\end{itemize*}

\subsection*{transverse waves}
\textcolor{magenta}{(from wikipedia:)}
moving waves that \emph{consist} of oscillations occurring perpendicular
to the direction of energy transfer.
If a \emph{transverse} wave is moving in the positive x-direction,
its \emph{oscillations} are in up and down directions that lie in the y–z plane.
Light is an example of a transverse wave.
($\vec B$ and $\vec E$ oscillate in directions perpendicular to the direction
in which the light is actually traveling).
With regard to transverse waves in matter,
the \emph{displacement of the medium} is perpendicular to the
direction of propagation of the wave.
Examples: A ripple in a pond and a wave on a string.

\subsection*{trapped modes}
For a specified geometry, uniqueness of the solution to a forcing problem
at a particular frequency is equivalent to the non-existence of a trapped
mode at that frequency. A trapped mode is a solution of the corresponding
homogeneous problem and represents a free oscillation with finite energy
of the fluid surrounding the fixed structure. For a given structure,
trapped modes may exist only at discrete frequencies.
Mathematically, a trapped mode corresponds to an eigenvalue embedded
in the continuous spectrum of the relevant operator.

\section*{U}
\section*{W}

\subsection*{waves}
\begin{itemize*}
    \item Caused by any turbulence in a medium.
    \item Observable properties
        \begin{itemize*}
            \item Period
            \item wavelength
            \item amplitude
            \item temporal/spatial signatures (shape of perturbation)
            \item characteristic scenerios of evolution (e.g.\ damped?)
        \end{itemize*}
\end{itemize*}

\subsection*{waveguide}
\begin{itemize*}
    \item Width $\sim$ same order of magnitude as the wavelength of
        the guided wave.
    \item Pores extending up from the photosphere
        into the solar atmosphere can act as an
        MHD waveguide.
    \item Flux tubes are excellent waveguides
\end{itemize*}

\subsection*{wave number}
Number of waves in a unit distance.
$ k = 2\pi $

\section*{X}
\section*{Y}
\section*{Z}

\end{document}
