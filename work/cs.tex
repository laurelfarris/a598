\documentclass[12pt]{article}
\usepackage[margin=1in]{geometry}
\usepackage{titlesec}
\setlength{\parindent}{0em}
\setlength{\parskip}{0.5em}
\usepackage{enumitem}
\usepackage{mdwlist}
\usepackage{color}

\title{\vspace{-0.5in}Wave Stuff}
\author{}
\date{}

\begin{document}
\maketitle

\vspace{-1in}

\paragraph{acoustic waves}
\begin{itemize*}
    \item Longitudinal
    \item propagate by means of adiabatic compression and decompression
    \item travel at sound speed (determined by medium)
    \item Important quantities:
        \begin{itemize*}
            \item sound pressure
            \item particle velocity
            \item particle displacement
            \item sound intensity
        \end{itemize*}
    \item exist because of a pressure restoring force: local compression
        (or rarefaction) sets up a \emph{pressure gradient} in opposition
        to the motion
    \item carry energy away from source
    \item large enough amplitude $\rightarrow$ shock wave,
        but usually small; ambient gas slightly disturbed
    \item exist in medium with low or non-existent magnetic field
    \item isotropic
\end{itemize*}

\paragraph{Alfv\'en waves}
Incompressible,

\paragraph{body modes}


\paragraph{dispersion relations}
Equations that describe

\paragraph{fast waves}
$C_{A_0} < C_{fast} < C_{A_e} $
Fast modes are highly dispersive.

\paragraph{gravity waves}

\paragraph{kink modes/waves/oscillations}
\textcolor{magenta}{(from wikipedia:)}
Kink (or transverse) modes,
are oblique (inclined with respect to the flow direction)
fast magnetoacoustic
(also known as magnetosonic waves) guided by the plasma structure;
the mode \emph{causes the displacement of the axis of the plasma structure}.
These modes are \emph{weakly compressible}, but could nevertheless be
observed with imaging instruments as \emph{periodic standing}
or \emph{propagating} displacements of coronal structures, e.g.\ coronal loops.
The frequency of transverse or ``kink'' modes is given by the following expression:
    $$ w_K = \sqrt{ \frac{2k_zB^2}{\mu(\rho_i+\rho_e)}  }   $$
In a cylindrical model of a loop,
the parameter \emph{azimuthal wave number},
m, is equal to 1 for kink modes.
This means that the cylinder is ``swaying with fixed ends''$\ldots$?

In the long wavelength limit, the phase speed of all but
sausage fast modes tends to the so-called kink speed,
which corresponds to the density weighed average Alfv\'en speed.

\paragraph{leaky modes}
Waves are allowed to radiate into the external medium, i.e.\
the condition of mode localization is relaxed.
Bessel functions are replaced by Hankel functions in the
dispersion relation.

\paragraph{longitudinal waves}
waves in which the displacement of the \emph{medium} is in the
same direction as, or the opposite direction to,
the direction of travel of the wave.

\paragraph{magnetoacoustic waves}
A magnetosonic wave (also magnetoacoustic wave) is a longitudinal wave
of ions (and electrons) in a magnetized plasma propagating perpendicular
to the stationary magnetic field.
Magnetoacoustic modes are collectively supported by the plasma
environment., i.e., the wave mode acts across neighbouring magnetic field
lines and across transverse plasma inhomogeneities.

\paragraph{modes}
A wave may be a superposition of lots of other waves. Each of those
waves is a ``mode'' of the resultant wave (think of the foundation
of Fourier Analysis: sums of sines and cosines).
Modes with the lowest wave number are \emph{global}, or
\emph{fundamental} modes.


\paragraph{oscillations}
Three types:
\begin{enumerate*}
    \item un-damped
    \item damped
    \item forced
\end{enumerate*}

\paragraph{phase mixing}

\textbf{pressure waves}
This is what pressure waves are $\ldots$

\paragraph{resonant absorption}

\paragraph{sausage modes}
The sausage mode approaches a cut-off at the external Alfv\'en speed.
Trapped sausage modes do not exist at longer wavelengths.
Mainly transverse. Another fast, localized, magnetoacoustic wave.

\paragraph{speeds}
Characteristic speeds of MHD are the sound speed and Alfv\'enic speed,
given by:
$ C_s = \sqrt{\frac{\gamma p_0}{\rho_0}} $ and
$ C_A = \frac{B_0}{\sqrt{\mu_0\rho_0}} $


\paragraph{slow waves}

\paragraph{surface modes}
evanescent behavior in both media (inside and outside cylindar).
        $$C_{T_0} < C_{slow} < C_{s_0} $$

\paragraph{transverse waves}
\textcolor{magenta}{(from wikipedia:)}
moving waves that \emph{consist} of oscillations occurring perpendicular
to the direction of energy transfer.
If a \emph{transverse} wave is moving in the positive x-direction,
its \emph{oscillations} are in up and down directions that lie in the y–z plane.
Light is an example of a transverse wave.
($\vec B$ and $\vec E$ oscillate in directions perpendicular to the direction
in which the light is actually traveling).
With regard to transverse waves in matter,
the \emph{displacement of the medium} is perpendicular to the
direction of propagation of the wave.
Examples: A ripple in a pond and a wave on a string.

\paragraph{waves}
Caused by any turbulence in a medium.

\paragraph{wave number}
Number of waves in a unit distance.
$ k = 2\pi $

\end{document}
