\documentclass[preprint2]{aastex}
\usepackage{natbib}
\usepackage{booktabs}
\bibliographystyle{apj}

%\tabletypesize{\small}
%\rotate
%\tablewidth{dimen} % Default - pagewidth
%\tablenum{1}
%\tablecolumns{num}
%\tablecaption{Characteristic properties of the basic MHD waves,
%    along with their observational techniques\label{prop}}
%\tablehead{\colhead{} & \colhead{} & \colhead{}}

\shorttitle{Coronal Seismology}
\shortauthors{Farris}

\begin{document}

\title{\vspace{-0.75in}Coronal Seismology}
\author{\vspace{-0.25in}Laurel Farris}
\affil{New Mexico State University}
\email{laurel07@nmsu.edu}

\begin{abstract}
Coronal seismology involves the investigation of magnetohydrodynamic
(MHD) waves and
ocillatory phenomenae that arise in the solar corona. Here some of the
dominant waves, oscillations, and modes are intimately investigated in
the literature. Analysis of data from the Atmospheric Imaging
Assembly (AIA) instrument on the Solar Dynamics Observatory (SDO) is
also presented, both as stand-alone research and in the broader
context of coronal seismology.
\end{abstract}
\keywords{Sun: corona \- Sun: oscillations \- Sun: seismology}

\section{Introduction}\label{intro}

The basic concepts of MHD waves and oscillations are described in
\S\ref{MHD}, and
\S\ref{topics} covers several MHD modes in detail.
Many of the techniques employed in coronal seismology were applied to AIA
data; a description of the project and the results are presented in
\S\ref{data} and \S\ref{analysis}.
Conclusions and future work are discussed in
\S\ref{conclusion}.
%====================================================================%
\section{Basic MHD}\label{MHD}
%--------------------------------------------------------------------%
\subsection{Types of Waves}
In general, MHD waves are divided into two categories:
Alfv\'en waves and magnetoacoustic waves.
Magnetoacoustic waves are further subdivided into
slow-mode and fast-mode (\cite{Asc}).
The slow-mode waves have phase speeds roughly equal to the sound speed
in the medium in which they reside, analogous to typical acoustic waves,
or sound waves. MHD studies usually focus on the fast-mode.
Fast-mode magnetoacoustic waves have phase speeds comparable to the Alfv\'en
speed, or $\omega/k \approx v_A$ (\cite{kink_1}).
Two common fast modes in the corona are the asymmetric \emph{kink}
oscillations and symmetric \emph{sausage} oscillations,
which are discussed in more detail
in \S{\ref{kink}} and \S{\ref{sausage}}, respectively.

The categories of MHD waves and oscillations can be best understood
by their relative speeds. Figure~\ref{speeds} shows the wave speeds relative
to the internal and external Alfv\'en and sound speeds.

\begin{figure}[ht]
    \plotone{disp_diagram.png}
    \caption{text (Image credit:~\cite{Nak})}
    \label{speeds}
\end{figure}

The sound speed in a medium is determined by thermal properties:
the thermal pressure and the mass density of the medium.
It is given by:
\begin{equation}\label{sound_speed}
    c_s = \sqrt{\frac{\gamma }{\rho}}
\end{equation}
The Alfv\'en speed, on the other hand, is determined by the magnetic
pressure:
\begin{equation}\label{Alfven_speed}
    c_A = \frac{B}{\sqrt{4\pi\rho}}
\end{equation}
Each MHD mode is characterized based on a number of qualities, such
as its wavelength, period, lifetime, speed, and
whether it is propagting or standing, fast or slow, longitudinal
or transverse, etc. The driving mechanisms that give rise to each mode
can differ depending on these qualities and the environment in which
the modes reside, and is one of the important MHD topics under investigation,
along with the damping mechanism (\cite{kink_1}).

%--------------------------------------------------------------------%
\subsection{Equations and Models}
MHD waves are often modeled with a cylindrical flux tube embedded
in a magnetic field.

[cylindrical model, equations of ideal MHD, etc.]
\begin{equation}
 \xi(x) = \xi(r)e^{i(kx+m\phi)}
\end{equation}
For kink oscillations, m=1, and for sausage modes, m=0.

There are several equations for ideal MHD from which the dispersion
relations are derived (I think).
%--------------------------------------------------------------------%
\subsection{Excitation and Damping Mechanisms}
\subsubsection{Resonant Absorption}
\subsubsection{Phase Mixing}
%--------------------------------------------------------------------%
\subsection{Observation Techniques}
Flux tubes (coronal loops), doppler shift and intensity variations,
density variations, velocity and magnetic field values,
etc. Coronal loops are density inhomogeneities in the corona,
where the (low-$\beta$) plasma is ``frozen-in'' along the magnetic
field lines whose feet are anchored in the photosphere.
%====================================================================%
\section{Types of MHD Modes}\label{topics}
%--------------------------------------------------------------------%
\subsection{Kink Oscillations}\label{kink}
Kink oscillations are directly observed in coronal loops in extreme
ultraviolet (EUV) wavelengths.
They characterize the spacial oscillations that occur over the surface of
the loop, perpendicular to the direction traversed by the length
of the loop (\cite{Nak}).
Kink oscillations generally are not accompanied by intensity variations;
they displace the axis of the magnetic structure in which they propagate,
but the cross-section of the waveguide remains roughly the same.
Kink oscillations occur in the ``long-wavelength regime'',
which is defined by
the product of the wavenumber and the cross-section of the coronal
loop being much less than 1 ($ka \ll 1$). In other words, the
\emph{wavelength} of the oscillation is much \emph{greater} than the
cross-section of the waveguide.
The phase speed is just above the Alfv\'en speed within the loop,
and the period of the oscillation is expected to be between
$\sim$ 2 and 20 minutes (\cite{Asc}).

Due to a lack of sufficient instrumentation,
spatial variations such as those caused by kink oscillations
were not resolved until the launch of TRACE\@.
Some of the first results from preliminary TRACE data were produced by
\cite{kink_1} to investigate the oscillations present in coronal loops.
Using 171 \AA{} data, they modeled five loops
that accompanied a class M4.6 solar flare in July of 1998.
At this point, many MHD modes were characterized in theory

The observations had enough qualities characteristic of
fast kink modes, including the spatial displacements characterisic
of this asymmetrical mode, to identify these oscillations as kink
modes.
An average period of $T = 269 \pm 6$ seconds, or
roughly 4.5 minutes, was obtained, which fit well with the theoretical
period for kink oscillations.
The absense of any phase
shift along the length of the loops revealed that these were
\emph{standing waves}, with nodes located at the loop footpoints.
The ability to identify kink modes has great significance
for solar phsyics due to a correlation between the oscillation period,
$T$, and the magnetic field strength, $B$, of the loop:
\begin{equation}
    T \propto \frac{L}{\sqrt{n}}{B}
\end{equation}
where $L$ is the loop length and $n$ is the number density,
both of which can be estimated from known coronal conditions.

%----paper2----%
More recently,~\cite{kink_2} investigated the driving mechanism
behind the production, and damping of kink oscillations.
They compared two possible functional form of the damping profile
of the driver: that of a Gaussian and an exponential form.
While the noise level of the data was too high to distinguish
between the two forms, the simulations followed the form of a
Gaussian.

They also considered the effect of the spatial profile of the driver
itself on the excitation and subsequent damping of the kink
waves. Two different possiblilities were explored here:
the effect of a ``highly structured'' driver, which they
found to be unrealistic, and the effects of eddies and photospheric
motions around the footpoints of the coronal loops.

\subsection{Sausage Oscillations}\label{sausage}
In contrast to kink oscillations, sausage oscillations do not displace
the axis of the structure in which they reside. However, they do cause
a periodic change in the cross-section of the waveguide, and hence
are observable through changes in intensity (and therefore density,
due to flux conservation).

\cite{sausage_1} plotted the changes in intensity and cross-sectional
area for sausage oscillations in photospheric pores extending up
through the solar atmosphere. They used the general cylinder model for
the pores, though it is more likely that the cross-sectional area of
the waveguide increases with height as the temperature increases and
density decreases.
Sausage waves are characterized by a change in the cross-section,
but no displacement of the loop axis. They occur on much shorter
timescales than kink waves.
The relationship between pore size and intensity can
indicate


\subsection{Acoustic Oscillations}

The relatives periodicities of aacoustic oscillations were recognized
as important characteristics, and modelled by~\cite{acoustic_1}.
They recognized magnetoacoustic oscillations as a useful way to determine
other properties of the solar atmosphere.

\cite{acoustic_2} used intensity oscillations of a few different ionization
species to pinpoint the orgin and progression of magnetoacoustic waves
between bright points in the photosphere, through the transition regions,
and up into the corona. The time series of intensity oscillations was
converted into a power spectrum, and periodicities were extracted using
wavelet analysis and periodograms (note here on what exactly these are?).
The periods they derived
($\sim 263 \pm 80$ s for the He II 256.32 \AA{} emission line and
($\sim 241 \pm 60$ s for the Fe XII 195.12 \AA{}  emission line)
were close to the 5-minute global oscillations of the sun.



%-------------------------------------------------------------------%
\subsection{Propagating Acoustic Waves}
The MHD modes that have been discussed up to now are considered
\emph{standing} modes: waves with nodes in a fixed
position, such as the footpoints of coronal loops that are anchored in
the photosphere. These are referred to as \emph{oscillations}.
In contrast, \emph{propagating} waves have nodes that move with time
(though two propagating waves traveling in opposite directions can
have a net result of a single, standing oscillation; see
\cite{Asc}, ch. 8).

Standing acoustic waves are primarily seen in closed coronal
loops, while propagating waves have been observed in both closed 
and open structures.
Waves with propagations speed much lower than the 
local Alfv\'en speed, are categorized as slow magnetoacoustic waves.
These waves travel along magnetic field lines at speeds roughly equal to
the local sound speed.
The changes in intensity along the same location as these waves
propagate in time are mapped side by side to give time-distance
information. The period, wavelength, propagation speed, and amplitude
can all be derived using this method.
Since the local sound, $c_{s}$ is related to temperature, $T$, as
\begin{equation}\label{sound-temp}
 c_{s} \propto \sqrt{T}
\end{equation}
a difference in observed propagation speeds implies a changing
temperature profile in the transverse direction across the loop.
(\cite{Nak}).

Propagating acoustic waves are generally observed as ``propagating
disturbances'' awaiting further diagnoses of magnetoacoustic waves,
whose amplitudes are rather weak, requiring careful, detailed data
analysis to aquire good signal-to-noise (S/N).

One of the first studies to analyze
simultaneous observations at different wavelengths was carried out by
\cite{pac_1} using data from two different instruments:
the Extreme ultraviolet Imaging Telescope (EIS)
on the Solar and Heliospheric Observatory (SOHO) and the
Transition Region And Coronal Explorer (TRACE).
These wavelengths, along with their corresponding ions emitting at those
wavelengths, the temperature, and other relevant quantities from the
study are given in table \ref{stuff}.
%-------------------------------------%
\begin{table}[h]
\centering
\begin{tabular}{c c c}
\hline\hline
instrument & EIT & TRACE\\
\hline
ion & Fe {\footnotesize XII} & Fe {\footnotesize IX}\\
wavelength & 195 \AA{} & 171 \AA{}\\
cadence & 15 s & 25 s\\
temperature & 1.6 MK & 1 MK\\
sound speed & 192 km s$^{-1}$ & 152 km s$^{-1}$\\
propagation speed & 110 km s$^{-1}$ & 95 km s$^{-1}$\\
\hline\hline
\end{tabular}
\caption{Relevant quantities from \cite{pac_1}}
\label{stuff}
\end{table}
%-------------------------------------%
Both instruments observed AR 8218, where the presence of
``weak transient disturbances'' were revealed, and later classified as
slow, propagating magnetoacoustic waves, with speeds that varied
between 65 and 150 km s$^{-1}$.
The expression for the formal sound speed of the ambient region
was given by
\begin{equation}
    c_{s} = 152\ \sqrt{T}\ \textrm{m}\ \textrm{s}^{-1}
\end{equation}
where $T$ is in units of Kelvin.
This was compared to the observed speed of the progating wave, which
was derived from the slope of the ``ridges'' on each of the four
time-distance plots.
The propatation speed of each wave was slightly ($\sim$ same order of
magnitude) lower than the local sound speed. This difference
was due to the angle between the loop and the plane of the sky against
which the observations were made.
The amount of time for a disturbance to pass a particular point was
determined to be $\sim$ 169 s from the TRACE data.
As both observations targeted the same wave, the signficance of the
differing speeds for each observation indicated a temperature gradient
in the loop itself, indicating either a bundle of loop threads that
make up a single loop, or a number of concentric shells that make up a
single loop.

A similar analysis in multiple wavelengths was carried out by
\cite{pac_2}, using about 6 hours of observational data from the
Atmospheric Imaging Assemply (AIA) on the Solar Dynamics Observatory (SDO).
In addition to these observations, they introduced the
``surfacing transform technique''
in order to address the difficulty in extracting information from these
low-amplitude waves amidst the noise (a typical signal-to-noise
($S/N$) ratio was $\sim$ 0.1 \%). (more information about this
technique here?)
This method involved the resonant behavior of ``surfing signals'',
revealed during quasi-periodic disturbances.
%Each surfing signal
%(ST) is a periodic disturbance as a function of time and frequency

This technique was applied to the lines at
131 \AA{}, 171 \AA{}, 193 \AA{}, and 211 \AA{},
over the active region (AR) NOAA AR 11082, where no sunspots were seen
to be present and the flares were relatively low-energy.
They found the same relationship between sound speed and temperature
as shown in equation \ref{sound-temp}, though the driving mechanism
for these disturbances was uncertain. The observed propagations were
travelling primarily in the upward direction away from the footpoint,
at speeds of about 40 to 180 km s$^{-1}$ and periodicities between 4
and 8 minutes over all four channels.


\subsection{Propagating Fast Waves}


\subsection{Torsional Modes}
Torsional modes, or more commonly, \emph{Alfv\'en} modes, are axisymmetric
modes whose speed is determined purely by magnetic pressure forces
(see equation \ref{Alfven_speed}).
They are transverse waves that are highly anisotropic (\cite{intro}),
traveling strictly along magnetic field lines in response
to a restoring force that acts to resist shear (cite class notes?).

In the classic cylindrical model of a plasma structure,
Alfv\'en waves undergo torsion (aka, twisting), in opposite directions on
either end of the cylindar (\cite{Nak}).

\subsection{Mixed Modes}




\section{Discussion}\label{disc}
The observational techniques and relevant properties of each of the different
kinds of MHD waves are summarized in table 1.
%----table----%
\begin{table*}[ht]
    \centering
    \begin{tabular}{c c c c}
        \toprule
        Type of wave &
        Timescales &
        Sizescales &
        Observational techniques\\
        \midrule
        Kink Oscillations & short & short & something\\
        Sausage Oscillations & short & short & something\\
        Acoustic Oscillations & x & x & x\\
        Propagating & x & x & x\\
        \bottomrule
    \end{tabular}
\end{table*}
%-------------%

\section{Data}\label{data}
As part of the general topic of coronal seismology,
a small research project was carried out as well, continuing
over from several semesters previously. Several of the observational
and analytical methods used in the literature were reproduced for
this project.

\section{Analysis}\label{analysis}

\section{Conclusion}\label{conclusion}
And we're finished.

\bibliography{reffile}
\end{document}
