\documentclass[preprint2]{aastex}
\usepackage{natbib}
\bibliographystyle{apj}

\newcommand{\mhd}{magnetohydrodynamic}

\shorttitle{Coronal Seismology}
\shortauthors{Farris}

\begin{document}

\title{\vspace{-0.75in}Coronal Seismology}
\author{\vspace{-0.25in}Laurel Farris}
\affil{New Mexico State University}
\email{laurel@laurelfarris.com}

\begin{abstract}
    Coronal seismology involves the investigation of {\mhd} (MHD) waves and
ocillatory phenomenae that arise in the solar corona. Here some of the
dominant waves, oscillations, and modes are intimately investigated in
the literature. Analysis of data from the Atmospheric Imaging
Assembly (AIA) instrument on the Solar Dynamics Observatory (SDO) is
also presented, both as stand-alone research and in the broader
context of coronal seismology.
\end{abstract}
\keywords{Sun: corona \- Sun: oscillations \- Sun: seismology}

\section{Introduction}\label{intro}
\subsection{Motivation}
[Motivation: coronal heating, other questions$\ldots$]

\subsection{MHD}
The heating of the corona by {\mhd} (MHD) waves is one of two prevalent theories
on how it reaches such high temperatures, the other being magnetic
reconnection (\cite{acoustic_1}).

In \S\ref{MHD}, several major types of waves and oscillatory modes in
the solar corona are described, along with recent investigations into
each one. \S\ref{data} includes a description of a research project and
its implications for the broader field of coronal seismology in
\S\ref{analysis}. Conclusions and future work are summed up in
\S\ref{conclusion}.


% General stuff about oscillations
\section{Oscillations}\label{MHD}

$$ \xi(x) = \xi(r)e^{i(kx+m\phi)} $$

For kink oscillations, m=1, and for sausage
modes, m=0.

% Specific stuff about oscillations; referring to papers
\subsection{Kink Oscillations}
Kink oscillations are commonly associated with coronal loops, and
characterize the spacial oscillations that occur over the surface of
the loop (\cite{Nak}).


%While temporal variations have been observed for the past (?) years
%or so, spatial variations were not revealed until the launch of
%TRACE, which provided sufficient spatial resolution to distinguish
%the spatial displacements of MHD oscillations.
% or
%the different characteristics of MHD oscillations.

%----paper1----%
Some of the first observations of these spatial variations
were carried out by~\cite{kink_1},
who utilized some of the first data released from the TRACE mission in
order to investigate the oscillations present in coronal loops.
Using data taken with the 171\AA{} filter, they modeled five loops that
were present with a solar flare in 1998.
The resulting model had several qualities characteristic of
fast kink modes, including asymmetry
and
displacements represented by sine curves.
As the period of kink modes were already known to correlate
with the magnetic field strength of the loop, pinpointing this
type of mode as the driver in coronal loops provided a valuable
constraint on coronal conditions.
The absense of any phase
shift along the length of the loops revealed that these were
\emph{standing waves}, with nodes located at the loop footpoints.

%----paper2----%
More recently,~\cite{kink_2} investigated the driving mechanism
behind the production, and damping of kink oscillations.
They compared two possible functional form of the damping profile
of the driver: that of a Gaussian and an exponential form.
While the noise level of the data was too high to distinguish
between the two forms, the simulations followed the form of a
Gaussian.

They also considered the effect of the spatial profile of the driver
itself on the excitation and subsequent damping of the kink
waves. Two different possiblilities were explored here:
the effect of a ``highly structured'' driver, which they
found to be unrealistic, and the effects of eddies and photospheric
motions around the footpoints of the coronal loops.



\subsection{Sausage Oscillations}
\cite{sausage_1} plotted the changes in intensity and cross-sectional
area for sausage oscillations in photospheric pores extending up
through the solar atmosphere. They used the general cylinder model for
the pores, though it is more likely that the cross-sectional area of
the waveguide increases with height as the temperature increases and
density drops. The relationship between pore size and intensity can
indicate

\subsection{Acoustic Oscillations}
\section{Waves}
\subsection{Propagating Acoustic Waves}
\subsection{Propagating Fast Waves}
\section{Modes}
\subsection{Torsional Modes}
\subsection{Mixed Modes}

\section{Data}\label{data}
As part of the general topic of coronal seismology,
a small research project was carried out as well, continuing
over from several semesters previously.

\section{Analysis}\label{analysis}

\section{Conclusion}\label{conclusion}
And we're finished.

\bibliography{reffile}
\end{document}
